% Based on http://nbviewer.jupyter.org/github/ipython/nbconvert-examples/blob/master/citations/Tutorial.ipynb , authored by Brian E. Granger
% Default to the notebook output style

    


% Inherit from the specified cell style.




    
    % Declare the document class
    \documentclass[landscape,letterpaper,10pt,english]{article}

    
    
    \usepackage[T1]{fontenc}
    % Nicer default font (+ math font) than Computer Modern for most use cases
    \usepackage{mathpazo}

    % Basic figure setup, for now with no caption control since it's done
    % automatically by Pandoc (which extracts ![](path) syntax from Markdown).
    \usepackage{graphicx}
    % We will generate all images so they have a width \maxwidth. This means
    % that they will get their normal width if they fit onto the page, but
    % are scaled down if they would overflow the margins.
    \makeatletter
    \def\maxwidth{\ifdim\Gin@nat@width>\linewidth\linewidth
    \else\Gin@nat@width\fi}
    \makeatother
    \let\Oldincludegraphics\includegraphics
    % Set max figure width to be 80% of text width, for now hardcoded.
    \renewcommand{\includegraphics}[1]{\Oldincludegraphics[width=.8\maxwidth]{#1}}
    % Ensure that by default, figures have no caption (until we provide a
    % proper Figure object with a Caption API and a way to capture that
    % in the conversion process - todo).
    \usepackage{caption}
    \DeclareCaptionLabelFormat{nolabel}{}
    \captionsetup{labelformat=nolabel}

    \usepackage{adjustbox} % Used to constrain images to a maximum size 
    \usepackage{xcolor} % Allow colors to be defined
    \usepackage{enumerate} % Needed for markdown enumerations to work
    \usepackage{geometry} % Used to adjust the document margins
    \usepackage{amsmath} % Equations
    \usepackage{amssymb} % Equations
    \usepackage{textcomp} % defines textquotesingle
    % Hack from http://tex.stackexchange.com/a/47451/13684:
    \AtBeginDocument{%
        \def\PYZsq{\textquotesingle}% Upright quotes in Pygmentized code
    }
    \usepackage{upquote} % Upright quotes for verbatim code
    \usepackage{eurosym} % defines \euro
    \usepackage[mathletters]{ucs} % Extended unicode (utf-8) support
    \usepackage[utf8x]{inputenc} % Allow utf-8 characters in the tex document
    \usepackage{fancyvrb} % verbatim replacement that allows latex
    \usepackage{grffile} % extends the file name processing of package graphics 
                         % to support a larger range 
    % The hyperref package gives us a pdf with properly built
    % internal navigation ('pdf bookmarks' for the table of contents,
    % internal cross-reference links, web links for URLs, etc.)
    \usepackage{hyperref}
    \usepackage{longtable} % longtable support required by pandoc >1.10
    \usepackage{booktabs}  % table support for pandoc > 1.12.2
    \usepackage[inline]{enumitem} % IRkernel/repr support (it uses the enumerate* environment)
    \usepackage[normalem]{ulem} % ulem is needed to support strikethroughs (\sout)
                                % normalem makes italics be italics, not underlines
    \usepackage{mathrsfs}
    

    
    
    % Colors for the hyperref package
    \definecolor{urlcolor}{rgb}{0,.145,.698}
    \definecolor{linkcolor}{rgb}{.71,0.21,0.01}
    \definecolor{citecolor}{rgb}{.12,.54,.11}

    % ANSI colors
    \definecolor{ansi-black}{HTML}{3E424D}
    \definecolor{ansi-black-intense}{HTML}{282C36}
    \definecolor{ansi-red}{HTML}{E75C58}
    \definecolor{ansi-red-intense}{HTML}{B22B31}
    \definecolor{ansi-green}{HTML}{00A250}
    \definecolor{ansi-green-intense}{HTML}{007427}
    \definecolor{ansi-yellow}{HTML}{DDB62B}
    \definecolor{ansi-yellow-intense}{HTML}{B27D12}
    \definecolor{ansi-blue}{HTML}{208FFB}
    \definecolor{ansi-blue-intense}{HTML}{0065CA}
    \definecolor{ansi-magenta}{HTML}{D160C4}
    \definecolor{ansi-magenta-intense}{HTML}{A03196}
    \definecolor{ansi-cyan}{HTML}{60C6C8}
    \definecolor{ansi-cyan-intense}{HTML}{258F8F}
    \definecolor{ansi-white}{HTML}{C5C1B4}
    \definecolor{ansi-white-intense}{HTML}{A1A6B2}
    \definecolor{ansi-default-inverse-fg}{HTML}{FFFFFF}
    \definecolor{ansi-default-inverse-bg}{HTML}{000000}

    % commands and environments needed by pandoc snippets
    % extracted from the output of `pandoc -s`
    \providecommand{\tightlist}{%
      \setlength{\itemsep}{0pt}\setlength{\parskip}{0pt}}
    \DefineVerbatimEnvironment{Highlighting}{Verbatim}{commandchars=\\\{\}}
    % Add ',fontsize=\small' for more characters per line
    \newenvironment{Shaded}{}{}
    \newcommand{\KeywordTok}[1]{\textcolor[rgb]{0.00,0.44,0.13}{\textbf{{#1}}}}
    \newcommand{\DataTypeTok}[1]{\textcolor[rgb]{0.56,0.13,0.00}{{#1}}}
    \newcommand{\DecValTok}[1]{\textcolor[rgb]{0.25,0.63,0.44}{{#1}}}
    \newcommand{\BaseNTok}[1]{\textcolor[rgb]{0.25,0.63,0.44}{{#1}}}
    \newcommand{\FloatTok}[1]{\textcolor[rgb]{0.25,0.63,0.44}{{#1}}}
    \newcommand{\CharTok}[1]{\textcolor[rgb]{0.25,0.44,0.63}{{#1}}}
    \newcommand{\StringTok}[1]{\textcolor[rgb]{0.25,0.44,0.63}{{#1}}}
    \newcommand{\CommentTok}[1]{\textcolor[rgb]{0.38,0.63,0.69}{\textit{{#1}}}}
    \newcommand{\OtherTok}[1]{\textcolor[rgb]{0.00,0.44,0.13}{{#1}}}
    \newcommand{\AlertTok}[1]{\textcolor[rgb]{1.00,0.00,0.00}{\textbf{{#1}}}}
    \newcommand{\FunctionTok}[1]{\textcolor[rgb]{0.02,0.16,0.49}{{#1}}}
    \newcommand{\RegionMarkerTok}[1]{{#1}}
    \newcommand{\ErrorTok}[1]{\textcolor[rgb]{1.00,0.00,0.00}{\textbf{{#1}}}}
    \newcommand{\NormalTok}[1]{{#1}}
    
    % Additional commands for more recent versions of Pandoc
    \newcommand{\ConstantTok}[1]{\textcolor[rgb]{0.53,0.00,0.00}{{#1}}}
    \newcommand{\SpecialCharTok}[1]{\textcolor[rgb]{0.25,0.44,0.63}{{#1}}}
    \newcommand{\VerbatimStringTok}[1]{\textcolor[rgb]{0.25,0.44,0.63}{{#1}}}
    \newcommand{\SpecialStringTok}[1]{\textcolor[rgb]{0.73,0.40,0.53}{{#1}}}
    \newcommand{\ImportTok}[1]{{#1}}
    \newcommand{\DocumentationTok}[1]{\textcolor[rgb]{0.73,0.13,0.13}{\textit{{#1}}}}
    \newcommand{\AnnotationTok}[1]{\textcolor[rgb]{0.38,0.63,0.69}{\textbf{\textit{{#1}}}}}
    \newcommand{\CommentVarTok}[1]{\textcolor[rgb]{0.38,0.63,0.69}{\textbf{\textit{{#1}}}}}
    \newcommand{\VariableTok}[1]{\textcolor[rgb]{0.10,0.09,0.49}{{#1}}}
    \newcommand{\ControlFlowTok}[1]{\textcolor[rgb]{0.00,0.44,0.13}{\textbf{{#1}}}}
    \newcommand{\OperatorTok}[1]{\textcolor[rgb]{0.40,0.40,0.40}{{#1}}}
    \newcommand{\BuiltInTok}[1]{{#1}}
    \newcommand{\ExtensionTok}[1]{{#1}}
    \newcommand{\PreprocessorTok}[1]{\textcolor[rgb]{0.74,0.48,0.00}{{#1}}}
    \newcommand{\AttributeTok}[1]{\textcolor[rgb]{0.49,0.56,0.16}{{#1}}}
    \newcommand{\InformationTok}[1]{\textcolor[rgb]{0.38,0.63,0.69}{\textbf{\textit{{#1}}}}}
    \newcommand{\WarningTok}[1]{\textcolor[rgb]{0.38,0.63,0.69}{\textbf{\textit{{#1}}}}}
    
    
    % Define a nice break command that doesn't care if a line doesn't already
    % exist.
    \def\br{\hspace*{\fill} \\* }
    % Math Jax compatibility definitions
    \def\gt{>}
    \def\lt{<}
    \let\Oldtex\TeX
    \let\Oldlatex\LaTeX
    \renewcommand{\TeX}{\textrm{\Oldtex}}
    \renewcommand{\LaTeX}{\textrm{\Oldlatex}}
    % Document parameters
    % Document title
    \title{Tutorial-GiRaFFE\_Higher\_Order}
    
    
%\author{Zachariah Etienne}
%
    

    % Pygments definitions
    
\makeatletter
\def\PY@reset{\let\PY@it=\relax \let\PY@bf=\relax%
    \let\PY@ul=\relax \let\PY@tc=\relax%
    \let\PY@bc=\relax \let\PY@ff=\relax}
\def\PY@tok#1{\csname PY@tok@#1\endcsname}
\def\PY@toks#1+{\ifx\relax#1\empty\else%
    \PY@tok{#1}\expandafter\PY@toks\fi}
\def\PY@do#1{\PY@bc{\PY@tc{\PY@ul{%
    \PY@it{\PY@bf{\PY@ff{#1}}}}}}}
\def\PY#1#2{\PY@reset\PY@toks#1+\relax+\PY@do{#2}}

\expandafter\def\csname PY@tok@w\endcsname{\def\PY@tc##1{\textcolor[rgb]{0.73,0.73,0.73}{##1}}}
\expandafter\def\csname PY@tok@c\endcsname{\let\PY@it=\textit\def\PY@tc##1{\textcolor[rgb]{0.25,0.50,0.50}{##1}}}
\expandafter\def\csname PY@tok@cp\endcsname{\def\PY@tc##1{\textcolor[rgb]{0.74,0.48,0.00}{##1}}}
\expandafter\def\csname PY@tok@k\endcsname{\let\PY@bf=\textbf\def\PY@tc##1{\textcolor[rgb]{0.00,0.50,0.00}{##1}}}
\expandafter\def\csname PY@tok@kp\endcsname{\def\PY@tc##1{\textcolor[rgb]{0.00,0.50,0.00}{##1}}}
\expandafter\def\csname PY@tok@kt\endcsname{\def\PY@tc##1{\textcolor[rgb]{0.69,0.00,0.25}{##1}}}
\expandafter\def\csname PY@tok@o\endcsname{\def\PY@tc##1{\textcolor[rgb]{0.40,0.40,0.40}{##1}}}
\expandafter\def\csname PY@tok@ow\endcsname{\let\PY@bf=\textbf\def\PY@tc##1{\textcolor[rgb]{0.67,0.13,1.00}{##1}}}
\expandafter\def\csname PY@tok@nb\endcsname{\def\PY@tc##1{\textcolor[rgb]{0.00,0.50,0.00}{##1}}}
\expandafter\def\csname PY@tok@nf\endcsname{\def\PY@tc##1{\textcolor[rgb]{0.00,0.00,1.00}{##1}}}
\expandafter\def\csname PY@tok@nc\endcsname{\let\PY@bf=\textbf\def\PY@tc##1{\textcolor[rgb]{0.00,0.00,1.00}{##1}}}
\expandafter\def\csname PY@tok@nn\endcsname{\let\PY@bf=\textbf\def\PY@tc##1{\textcolor[rgb]{0.00,0.00,1.00}{##1}}}
\expandafter\def\csname PY@tok@ne\endcsname{\let\PY@bf=\textbf\def\PY@tc##1{\textcolor[rgb]{0.82,0.25,0.23}{##1}}}
\expandafter\def\csname PY@tok@nv\endcsname{\def\PY@tc##1{\textcolor[rgb]{0.10,0.09,0.49}{##1}}}
\expandafter\def\csname PY@tok@no\endcsname{\def\PY@tc##1{\textcolor[rgb]{0.53,0.00,0.00}{##1}}}
\expandafter\def\csname PY@tok@nl\endcsname{\def\PY@tc##1{\textcolor[rgb]{0.63,0.63,0.00}{##1}}}
\expandafter\def\csname PY@tok@ni\endcsname{\let\PY@bf=\textbf\def\PY@tc##1{\textcolor[rgb]{0.60,0.60,0.60}{##1}}}
\expandafter\def\csname PY@tok@na\endcsname{\def\PY@tc##1{\textcolor[rgb]{0.49,0.56,0.16}{##1}}}
\expandafter\def\csname PY@tok@nt\endcsname{\let\PY@bf=\textbf\def\PY@tc##1{\textcolor[rgb]{0.00,0.50,0.00}{##1}}}
\expandafter\def\csname PY@tok@nd\endcsname{\def\PY@tc##1{\textcolor[rgb]{0.67,0.13,1.00}{##1}}}
\expandafter\def\csname PY@tok@s\endcsname{\def\PY@tc##1{\textcolor[rgb]{0.73,0.13,0.13}{##1}}}
\expandafter\def\csname PY@tok@sd\endcsname{\let\PY@it=\textit\def\PY@tc##1{\textcolor[rgb]{0.73,0.13,0.13}{##1}}}
\expandafter\def\csname PY@tok@si\endcsname{\let\PY@bf=\textbf\def\PY@tc##1{\textcolor[rgb]{0.73,0.40,0.53}{##1}}}
\expandafter\def\csname PY@tok@se\endcsname{\let\PY@bf=\textbf\def\PY@tc##1{\textcolor[rgb]{0.73,0.40,0.13}{##1}}}
\expandafter\def\csname PY@tok@sr\endcsname{\def\PY@tc##1{\textcolor[rgb]{0.73,0.40,0.53}{##1}}}
\expandafter\def\csname PY@tok@ss\endcsname{\def\PY@tc##1{\textcolor[rgb]{0.10,0.09,0.49}{##1}}}
\expandafter\def\csname PY@tok@sx\endcsname{\def\PY@tc##1{\textcolor[rgb]{0.00,0.50,0.00}{##1}}}
\expandafter\def\csname PY@tok@m\endcsname{\def\PY@tc##1{\textcolor[rgb]{0.40,0.40,0.40}{##1}}}
\expandafter\def\csname PY@tok@gh\endcsname{\let\PY@bf=\textbf\def\PY@tc##1{\textcolor[rgb]{0.00,0.00,0.50}{##1}}}
\expandafter\def\csname PY@tok@gu\endcsname{\let\PY@bf=\textbf\def\PY@tc##1{\textcolor[rgb]{0.50,0.00,0.50}{##1}}}
\expandafter\def\csname PY@tok@gd\endcsname{\def\PY@tc##1{\textcolor[rgb]{0.63,0.00,0.00}{##1}}}
\expandafter\def\csname PY@tok@gi\endcsname{\def\PY@tc##1{\textcolor[rgb]{0.00,0.63,0.00}{##1}}}
\expandafter\def\csname PY@tok@gr\endcsname{\def\PY@tc##1{\textcolor[rgb]{1.00,0.00,0.00}{##1}}}
\expandafter\def\csname PY@tok@ge\endcsname{\let\PY@it=\textit}
\expandafter\def\csname PY@tok@gs\endcsname{\let\PY@bf=\textbf}
\expandafter\def\csname PY@tok@gp\endcsname{\let\PY@bf=\textbf\def\PY@tc##1{\textcolor[rgb]{0.00,0.00,0.50}{##1}}}
\expandafter\def\csname PY@tok@go\endcsname{\def\PY@tc##1{\textcolor[rgb]{0.53,0.53,0.53}{##1}}}
\expandafter\def\csname PY@tok@gt\endcsname{\def\PY@tc##1{\textcolor[rgb]{0.00,0.27,0.87}{##1}}}
\expandafter\def\csname PY@tok@err\endcsname{\def\PY@bc##1{\setlength{\fboxsep}{0pt}\fcolorbox[rgb]{1.00,0.00,0.00}{1,1,1}{\strut ##1}}}
\expandafter\def\csname PY@tok@kc\endcsname{\let\PY@bf=\textbf\def\PY@tc##1{\textcolor[rgb]{0.00,0.50,0.00}{##1}}}
\expandafter\def\csname PY@tok@kd\endcsname{\let\PY@bf=\textbf\def\PY@tc##1{\textcolor[rgb]{0.00,0.50,0.00}{##1}}}
\expandafter\def\csname PY@tok@kn\endcsname{\let\PY@bf=\textbf\def\PY@tc##1{\textcolor[rgb]{0.00,0.50,0.00}{##1}}}
\expandafter\def\csname PY@tok@kr\endcsname{\let\PY@bf=\textbf\def\PY@tc##1{\textcolor[rgb]{0.00,0.50,0.00}{##1}}}
\expandafter\def\csname PY@tok@bp\endcsname{\def\PY@tc##1{\textcolor[rgb]{0.00,0.50,0.00}{##1}}}
\expandafter\def\csname PY@tok@fm\endcsname{\def\PY@tc##1{\textcolor[rgb]{0.00,0.00,1.00}{##1}}}
\expandafter\def\csname PY@tok@vc\endcsname{\def\PY@tc##1{\textcolor[rgb]{0.10,0.09,0.49}{##1}}}
\expandafter\def\csname PY@tok@vg\endcsname{\def\PY@tc##1{\textcolor[rgb]{0.10,0.09,0.49}{##1}}}
\expandafter\def\csname PY@tok@vi\endcsname{\def\PY@tc##1{\textcolor[rgb]{0.10,0.09,0.49}{##1}}}
\expandafter\def\csname PY@tok@vm\endcsname{\def\PY@tc##1{\textcolor[rgb]{0.10,0.09,0.49}{##1}}}
\expandafter\def\csname PY@tok@sa\endcsname{\def\PY@tc##1{\textcolor[rgb]{0.73,0.13,0.13}{##1}}}
\expandafter\def\csname PY@tok@sb\endcsname{\def\PY@tc##1{\textcolor[rgb]{0.73,0.13,0.13}{##1}}}
\expandafter\def\csname PY@tok@sc\endcsname{\def\PY@tc##1{\textcolor[rgb]{0.73,0.13,0.13}{##1}}}
\expandafter\def\csname PY@tok@dl\endcsname{\def\PY@tc##1{\textcolor[rgb]{0.73,0.13,0.13}{##1}}}
\expandafter\def\csname PY@tok@s2\endcsname{\def\PY@tc##1{\textcolor[rgb]{0.73,0.13,0.13}{##1}}}
\expandafter\def\csname PY@tok@sh\endcsname{\def\PY@tc##1{\textcolor[rgb]{0.73,0.13,0.13}{##1}}}
\expandafter\def\csname PY@tok@s1\endcsname{\def\PY@tc##1{\textcolor[rgb]{0.73,0.13,0.13}{##1}}}
\expandafter\def\csname PY@tok@mb\endcsname{\def\PY@tc##1{\textcolor[rgb]{0.40,0.40,0.40}{##1}}}
\expandafter\def\csname PY@tok@mf\endcsname{\def\PY@tc##1{\textcolor[rgb]{0.40,0.40,0.40}{##1}}}
\expandafter\def\csname PY@tok@mh\endcsname{\def\PY@tc##1{\textcolor[rgb]{0.40,0.40,0.40}{##1}}}
\expandafter\def\csname PY@tok@mi\endcsname{\def\PY@tc##1{\textcolor[rgb]{0.40,0.40,0.40}{##1}}}
\expandafter\def\csname PY@tok@il\endcsname{\def\PY@tc##1{\textcolor[rgb]{0.40,0.40,0.40}{##1}}}
\expandafter\def\csname PY@tok@mo\endcsname{\def\PY@tc##1{\textcolor[rgb]{0.40,0.40,0.40}{##1}}}
\expandafter\def\csname PY@tok@ch\endcsname{\let\PY@it=\textit\def\PY@tc##1{\textcolor[rgb]{0.25,0.50,0.50}{##1}}}
\expandafter\def\csname PY@tok@cm\endcsname{\let\PY@it=\textit\def\PY@tc##1{\textcolor[rgb]{0.25,0.50,0.50}{##1}}}
\expandafter\def\csname PY@tok@cpf\endcsname{\let\PY@it=\textit\def\PY@tc##1{\textcolor[rgb]{0.25,0.50,0.50}{##1}}}
\expandafter\def\csname PY@tok@c1\endcsname{\let\PY@it=\textit\def\PY@tc##1{\textcolor[rgb]{0.25,0.50,0.50}{##1}}}
\expandafter\def\csname PY@tok@cs\endcsname{\let\PY@it=\textit\def\PY@tc##1{\textcolor[rgb]{0.25,0.50,0.50}{##1}}}

\def\PYZbs{\char`\\}
\def\PYZus{\char`\_}
\def\PYZob{\char`\{}
\def\PYZcb{\char`\}}
\def\PYZca{\char`\^}
\def\PYZam{\char`\&}
\def\PYZlt{\char`\<}
\def\PYZgt{\char`\>}
\def\PYZsh{\char`\#}
\def\PYZpc{\char`\%}
\def\PYZdl{\char`\$}
\def\PYZhy{\char`\-}
\def\PYZsq{\char`\'}
\def\PYZdq{\char`\"}
\def\PYZti{\char`\~}
% for compatibility with earlier versions
\def\PYZat{@}
\def\PYZlb{[}
\def\PYZrb{]}
\makeatother


    % Exact colors from NB
    \definecolor{incolor}{rgb}{0.0, 0.0, 0.5}
    \definecolor{outcolor}{rgb}{0.545, 0.0, 0.0}



    
    % Prevent overflowing lines due to hard-to-break entities
    \sloppy 
    % Setup hyperref package
    \hypersetup{
      breaklinks=true,  % so long urls are correctly broken across lines
      colorlinks=true,
      urlcolor=urlcolor,
      linkcolor=linkcolor,
      citecolor=citecolor,
      }
    % Slightly bigger margins than the latex defaults
    
    \geometry{verbose,tmargin=1in,bmargin=1in,lmargin=1in,rmargin=1in}
    
    

    \begin{document}
    
    
    \maketitle
    
    

    
    \section{\texorpdfstring{\(\texttt{GiRaFFE}\): General Relativistic
Force-Free
Electrodynamics}{\textbackslash{}texttt\{GiRaFFE\}: General Relativistic Force-Free Electrodynamics}}\label{textttgiraffe-general-relativistic-force-free-electrodynamics}

\[\label{top}\]

\subsection{\texorpdfstring{Porting the original \(\texttt{GiRaFFE}\)
code to
NRPy+}{Porting the original \textbackslash{}texttt\{GiRaFFE\} code to NRPy+}}\label{porting-the-original-textttgiraffe-code-to-nrpy}

Porting the original \(\texttt{GiRaFFE}\) code, as presented in
\href{https://arxiv.org/pdf/1704.00599.pdf}{the original paper}, to
NRPy+ will generally make it easier to maintain, as well as to make
changes. Specifically, it will make it nearly trivial to increase the
finite-differencing order.

We will begin this tutorial in the usual way: import the core NRPy+
modules that we need, set the dimensionality of the grid with parameter
\(\text{grid::DIM}\), and declare the basic gridfunctions in
Section \ref{preliminaries}. We will then set
\(T^{\mu \nu}_{\rm EM} = b^2 u^{\mu} u^{\nu} + \frac{b^2}{2} g^{\mu \nu} - b^{\mu} b^{\nu}\)
and derive and set the related quantities
\(\partial_j T^{j}_{{\rm EM} i}\) in Section \ref{step4}. Then, in
Section \ref{step7}, we will use the previously constructed quantities
to build the right hand side for the time-evolution equation
\[\partial_t \tilde{S}_i = - \partial_j \left( \alpha \sqrt{\gamma} T^j_{{\rm EM} i} \right) + \frac{1}{2} \alpha \sqrt{\gamma} T^{\mu \nu}_{\rm EM} \partial_i g_{\mu \nu}.\]
After that, we will construct the evolution equations for the other
quantities in Section \ref{step8}
\[\partial_t A_i = \epsilon_{ijk} v^j B^k - \partial_i (\alpha \Phi - \beta^j A_j)\]
and
\[\partial_t [\sqrt{\gamma} \Phi] = -\partial_j (\alpha\sqrt{\gamma}A^j - \beta^j [\sqrt{\gamma} \Phi]) - \xi \alpha [\sqrt{\gamma} \Phi].\]
Finally, in Section \ref{step9}, we will build the expression
\[\tilde{S}_i = \gamma_{ij} \frac{v^i_{(n)} \sqrt{\gamma}B^2}{4 \pi}.\]
This will be needed to set initial data for \(\tilde{S}_i\) from the
initial data defined in the module \url{Tutorial-GiRaFFEfood_HO.ipynb}

\subsubsection{Table of Contents:}\label{table-of-contents}

\begin{enumerate}
\def\labelenumi{\arabic{enumi}.}
\tightlist
\item
  Preliminaries

  \begin{enumerate}
  \def\labelenumii{\arabic{enumii}.}
  \tightlist
  \item
    Section \ref{steps0to2}: Set up the basic NRPy+ infrastructure we
    need
  \item
    Section \ref{step3}: Build the spatial derivatives of the four
    metric
  \end{enumerate}
\item
  \(T^{\mu\nu}_{\rm EM}\) and its derivatives

  \begin{enumerate}
  \def\labelenumii{\arabic{enumii}.}
  \tightlist
  \item
    Section \ref{step4}: \(u^i\) and \(b^i\) and related quantities
  \item
    Section \ref{step5}: Construct the electromagnetic stress-energy
    tensor
  \item
    Section \ref{step6}: Derivatives of the electromagnetic
    stress-energy tensor
  \end{enumerate}
\item
  Evolution equation for \(\tilde{S}_i\)

  \begin{enumerate}
  \def\labelenumii{\arabic{enumii}.}
  \tightlist
  \item
    Section \ref{step7}: Construct the evolution equation for
    \(\tilde{S}_i\)
  \end{enumerate}
\item
  Evolution equations for \(A_i\) and \(\Phi\)

  \begin{enumerate}
  \def\labelenumii{\arabic{enumii}.}
  \tightlist
  \item
    Section \ref{step8}: Construct the evolution equations for \(A_i\)
    and \(\sqrt{\gamma}\Phi\)
  \end{enumerate}
\item
  Setting the initial \(\tilde{S}_i\) from initial data

  \begin{enumerate}
  \def\labelenumii{\arabic{enumii}.}
  \tightlist
  \item
    Section \ref{step9}: Build the expression for \(\tilde{S}_i\)
  \end{enumerate}
\item
  Code Validation

  \begin{enumerate}
  \def\labelenumii{\arabic{enumii}.}
  \tightlist
  \item
    Section \ref{step10}: NRPy+ Module Code Validation
  \end{enumerate}
\end{enumerate}

    Our ultimate goal here will be to code the evolution equations (from eq.
13 of the \href{https://arxiv.org/pdf/1704.00599.pdf}{original paper}):

\begin{align}
\partial_t \tilde{S}_i &= - \partial_j \left( \alpha \sqrt{\gamma} T^j_{{\rm EM} i} \right) + \frac{1}{2} \alpha \sqrt{\gamma} T^{\mu \nu}_{\rm EM} \partial_i g_{\mu \nu}\\
\partial_t A_i &= \epsilon_{ijk} v^j B^k - \partial_i (\alpha \Phi - \beta^j A_j)\\
\partial_t [\sqrt{\gamma} \Phi] &= -\partial_j (\alpha\sqrt{\gamma}A^j - \beta^j [\sqrt{\gamma} \Phi]) - \xi \alpha [\sqrt{\gamma} \Phi],
\end{align}

where the densitized spatial Poynting flux one-form
\(\tilde{S}_i = \sqrt{\gamma} S_i\) (and \(S_i\) comes from
\(S_{\mu} -n_{\nu} T^{\nu}_{{\rm EM} \mu}\)) and

\begin{align}
T^{\mu \nu}_{\rm EM} &= b^2 u^{\mu} u^{\nu} + \frac{b^2}{2} g^{\mu \nu} - b^{\mu} b^{\nu}, \\
\sqrt{4\pi} b^0 &= B^0_{\rm (u)} = \frac{u_j B^j}{\alpha}, \\
\sqrt{4\pi} b^i &= B^i_{\rm (u)} = \frac{B^i + (u_j B^j) u^i}{\alpha u^0}, \\
\end{align}

and \[
 B^i = \frac{\tilde{B}^i}{\sqrt{\gamma}}.\]

\(T^{\mu\nu}_{\rm EM}\) is written in terms of

\begin{itemize}
\tightlist
\item
  \(b^\mu\), the 4-component magnetic field vector, related to the
  comoving magnetic field vector \(B^i_{(u)}\)
\item
  \(u^\mu\), the 4-velocity
\item
  \(g^{\mu \nu}\), the inverse 4-metric
\end{itemize}

However, \(\texttt{GiRaFFE}\) has access to only the following
quantities

\begin{itemize}
\tightlist
\item
  \(\gamma_{ij}\), the 3-metric
\item
  \(\alpha\), the lapse
\item
  \(\beta^i\), the shift
\item
  \(A_i\), the vector potential
\item
  \(\left[\sqrt{\gamma}\Phi\right]\), the zero-component of the vector
  potential \(A_\mu\), times the square root of the determinant of the
  3-metric
\item
  \(v_{(n)}^i\), the Valencia 3-velocity
\item
  \(u^0\), the zero-component of the 4-velocity
\end{itemize}

Thus to compute the needed quantities appearing on the right-hand side
of \(\partial_t \tilde{S}_i\), \(\partial_t A_i\), and
\(\partial_t \left[\sqrt{\gamma}\Phi\right]\).

\begin{itemize}
\tightlist
\item
  \(b^\mu\):

  \begin{itemize}
  \tightlist
  \item
    \(b^0 = \frac{1}{\sqrt{4\pi}} B^0_{\rm (u)} = \frac{u_j B^j}{\sqrt{4\pi}\alpha}\)
  \item
    \(b^i = \frac{1}{\sqrt{4\pi}} B^i_{\rm (u)} = \frac{B^i + (u_j B^j) u^i}{\sqrt{4\pi}\alpha u^0}\)
  \item
    \(B^i = \epsilon^{ijk} \partial_j A_k\)
  \end{itemize}
\item
  \(u^\mu\):

  \begin{itemize}
  \tightlist
  \item
    \(u^i = u^0 (\alpha v^i_{(n)} - \beta^i)\)
  \end{itemize}
\item
  \(g^{\mu \nu}\):

  \begin{itemize}
  \tightlist
  \item
    \(g^{\mu\nu} = \begin{pmatrix} -\frac{1}{\alpha^2} & \frac{\beta^i}{\alpha^2} \\ \frac{\beta^i}{\alpha^2} & \gamma^{ij} - \frac{\beta^i\beta^j}{\alpha^2} \end{pmatrix},\)
  \end{itemize}
\end{itemize}

Most of these are computed in the module
u0\_smallb\_Poynting\_\_Cartesian.py, and we will import that module to
save effort. Furthermore, the four-metric \(g_{\mu\nu}\) is related to
the three-metric \(\gamma_{ij}\), index-lowered shift \(\beta_i\), and
lapse \(\alpha\) by \[
g_{\mu\nu} = \begin{pmatrix} 
-\alpha^2 + \beta^k \beta_k & \beta_j \\
\beta_i & \gamma_{ij}
\end{pmatrix}.
\] Note that as usual, Greek indices refer to four-dimensional
quantities where the zeroth component indicates \(t\) components, while
Latin indices refer to three-dimensional quantities. Since Python always
indexes its lists from 0, however, the zeroth component will indicate a
spatial direction, and any expressions involving mixed Greek and Latin
indices will need to offset one set of indices by one. To be more
specific, a Latin index in a four-vector must be incremented and a Greek
index in a three-vector must be decremented (however, this case does is
not as common, and does not occur in this tutorial).

    \section{Preliminaries}\label{preliminaries}

First, we will import the core modules of NRPy that we will need and
specify the main gridfunctions we will need.

\subsection{Steps 0-2: Set up the needed NRPy+
infrastructure}\label{steps-0-2-set-up-the-needed-nrpy-infrastructure}

\[\label{steps0to2}\]

{[}Back to Section \ref{top}{]}

\begin{enumerate}
\def\labelenumi{\arabic{enumi}.}
\setcounter{enumi}{-1}
\tightlist
\item
  Set the spatial dimension parameter to 3.
\item
  Set the finite differencing order to 4. The metric quantities
  \(\alpha\), \(\beta^i\), and \(\gamma_{ij}\) will be initially set by
  the \(\text{ShiftedKerrSchild}\) thorn, while the Valencia 3-velocity
  \(v^i_{(n)}\) and vector potential \(A_i\) will initially be set in
  the separate thorn \textbf{GiRaFFEfood\_HO}.
\item
  To import the four-metric, we will need to call the function
  **compute\_u0\_smallb\_Poynting\_\_Cartesian(gammaDD,betaU,alpha,ValenciavU,BU)\textbf{.
  Since it has BU as an argument, we will need to first use
  }ixp.symm\_matrix\_inverter3x3()** to find gammadet, which in turn
  allows us to set the Levi-Civita tensor, and then to construct BU.
\end{enumerate}

    \begin{Verbatim}[commandchars=\\\{\}]
{\color{incolor}In [{\color{incolor}1}]:} \PY{k+kn}{import} \PY{n+nn}{NRPy\PYZus{}param\PYZus{}funcs} \PY{k+kn}{as} \PY{n+nn}{par}
        \PY{k+kn}{import} \PY{n+nn}{indexedexp} \PY{k+kn}{as} \PY{n+nn}{ixp}
        \PY{k+kn}{import} \PY{n+nn}{grid} \PY{k+kn}{as} \PY{n+nn}{gri}
        \PY{k+kn}{import} \PY{n+nn}{finite\PYZus{}difference} \PY{k+kn}{as} \PY{n+nn}{fin}
        \PY{k+kn}{from} \PY{n+nn}{outputC} \PY{k+kn}{import} \PY{o}{*}
        
        \PY{c+c1}{\PYZsh{}Step 0: Set the spatial dimension parameter to 3.}
        \PY{n}{par}\PY{o}{.}\PY{n}{set\PYZus{}parval\PYZus{}from\PYZus{}str}\PY{p}{(}\PY{l+s+s2}{\PYZdq{}}\PY{l+s+s2}{grid::DIM}\PY{l+s+s2}{\PYZdq{}}\PY{p}{,} \PY{l+m+mi}{3}\PY{p}{)}
        \PY{n}{DIM} \PY{o}{=} \PY{n}{par}\PY{o}{.}\PY{n}{parval\PYZus{}from\PYZus{}str}\PY{p}{(}\PY{l+s+s2}{\PYZdq{}}\PY{l+s+s2}{grid::DIM}\PY{l+s+s2}{\PYZdq{}}\PY{p}{)}
        
        \PY{c+c1}{\PYZsh{} Step 1: Set the finite differencing order to 4.}
        \PY{n}{par}\PY{o}{.}\PY{n}{set\PYZus{}parval\PYZus{}from\PYZus{}str}\PY{p}{(}\PY{l+s+s2}{\PYZdq{}}\PY{l+s+s2}{finite\PYZus{}difference::FD\PYZus{}CENTDERIVS\PYZus{}ORDER}\PY{l+s+s2}{\PYZdq{}}\PY{p}{,} \PY{l+m+mi}{4}\PY{p}{)}
        
        \PY{n}{thismodule} \PY{o}{=} \PY{l+s+s2}{\PYZdq{}}\PY{l+s+s2}{GiRaFFE\PYZus{}NRPy}\PY{l+s+s2}{\PYZdq{}}
        
        \PY{c+c1}{\PYZsh{} M\PYZus{}PI will allow the C code to substitute the correct value}
        \PY{n}{M\PYZus{}PI} \PY{o}{=} \PY{n}{par}\PY{o}{.}\PY{n}{Cparameters}\PY{p}{(}\PY{l+s+s2}{\PYZdq{}}\PY{l+s+s2}{REAL}\PY{l+s+s2}{\PYZdq{}}\PY{p}{,}\PY{n}{thismodule}\PY{p}{,}\PY{l+s+s2}{\PYZdq{}}\PY{l+s+s2}{M\PYZus{}PI}\PY{l+s+s2}{\PYZdq{}}\PY{p}{)}
        \PY{c+c1}{\PYZsh{} Here, we declare the 3\PYZhy{}metric, shift, and lapse as usual}
        \PY{n}{gammaDD} \PY{o}{=} \PY{n}{ixp}\PY{o}{.}\PY{n}{register\PYZus{}gridfunctions\PYZus{}for\PYZus{}single\PYZus{}rank2}\PY{p}{(}\PY{l+s+s2}{\PYZdq{}}\PY{l+s+s2}{AUX}\PY{l+s+s2}{\PYZdq{}}\PY{p}{,}\PY{l+s+s2}{\PYZdq{}}\PY{l+s+s2}{gammaDD}\PY{l+s+s2}{\PYZdq{}}\PY{p}{,} \PY{l+s+s2}{\PYZdq{}}\PY{l+s+s2}{sym01}\PY{l+s+s2}{\PYZdq{}}\PY{p}{,}\PY{n}{DIM}\PY{o}{=}\PY{l+m+mi}{3}\PY{p}{)}
        \PY{n}{betaU}   \PY{o}{=} \PY{n}{ixp}\PY{o}{.}\PY{n}{register\PYZus{}gridfunctions\PYZus{}for\PYZus{}single\PYZus{}rank1}\PY{p}{(}\PY{l+s+s2}{\PYZdq{}}\PY{l+s+s2}{AUX}\PY{l+s+s2}{\PYZdq{}}\PY{p}{,}\PY{l+s+s2}{\PYZdq{}}\PY{l+s+s2}{betaU}\PY{l+s+s2}{\PYZdq{}}\PY{p}{,}\PY{n}{DIM}\PY{o}{=}\PY{l+m+mi}{3}\PY{p}{)}
        \PY{n}{alpha}   \PY{o}{=} \PY{n}{gri}\PY{o}{.}\PY{n}{register\PYZus{}gridfunctions}\PY{p}{(}\PY{l+s+s2}{\PYZdq{}}\PY{l+s+s2}{AUX}\PY{l+s+s2}{\PYZdq{}}\PY{p}{,}\PY{l+s+s2}{\PYZdq{}}\PY{l+s+s2}{alpha}\PY{l+s+s2}{\PYZdq{}}\PY{p}{)}
        \PY{c+c1}{\PYZsh{} These two gridfunctions are the basic inputs into StildeD\PYZus{}rhs}
        \PY{n}{ValenciavU} \PY{o}{=} \PY{n}{ixp}\PY{o}{.}\PY{n}{register\PYZus{}gridfunctions\PYZus{}for\PYZus{}single\PYZus{}rank1}\PY{p}{(}\PY{l+s+s2}{\PYZdq{}}\PY{l+s+s2}{AUX}\PY{l+s+s2}{\PYZdq{}}\PY{p}{,}\PY{l+s+s2}{\PYZdq{}}\PY{l+s+s2}{ValenciavU}\PY{l+s+s2}{\PYZdq{}}\PY{p}{,}\PY{n}{DIM}\PY{o}{=}\PY{l+m+mi}{3}\PY{p}{)}
        \PY{n}{AD} \PY{o}{=} \PY{n}{ixp}\PY{o}{.}\PY{n}{register\PYZus{}gridfunctions\PYZus{}for\PYZus{}single\PYZus{}rank1}\PY{p}{(}\PY{l+s+s2}{\PYZdq{}}\PY{l+s+s2}{AUX}\PY{l+s+s2}{\PYZdq{}}\PY{p}{,}\PY{l+s+s2}{\PYZdq{}}\PY{l+s+s2}{AD}\PY{l+s+s2}{\PYZdq{}}\PY{p}{,}\PY{n}{DIM}\PY{o}{=}\PY{l+m+mi}{3}\PY{p}{)}
        
        \PY{c+c1}{\PYZsh{} Step 2: Import the four metric}
        \PY{n}{gammaUU} \PY{o}{=} \PY{n}{ixp}\PY{o}{.}\PY{n}{register\PYZus{}gridfunctions\PYZus{}for\PYZus{}single\PYZus{}rank2}\PY{p}{(}\PY{l+s+s2}{\PYZdq{}}\PY{l+s+s2}{AUX}\PY{l+s+s2}{\PYZdq{}}\PY{p}{,}\PY{l+s+s2}{\PYZdq{}}\PY{l+s+s2}{gammaUU}\PY{l+s+s2}{\PYZdq{}}\PY{p}{,}\PY{l+s+s2}{\PYZdq{}}\PY{l+s+s2}{sym01}\PY{l+s+s2}{\PYZdq{}}\PY{p}{)}
        \PY{n}{gammadet} \PY{o}{=} \PY{n}{gri}\PY{o}{.}\PY{n}{register\PYZus{}gridfunctions}\PY{p}{(}\PY{l+s+s2}{\PYZdq{}}\PY{l+s+s2}{AUX}\PY{l+s+s2}{\PYZdq{}}\PY{p}{,}\PY{l+s+s2}{\PYZdq{}}\PY{l+s+s2}{gammadet}\PY{l+s+s2}{\PYZdq{}}\PY{p}{)}
        \PY{n}{gammaUU}\PY{p}{,} \PY{n}{gammadet} \PY{o}{=} \PY{n}{ixp}\PY{o}{.}\PY{n}{symm\PYZus{}matrix\PYZus{}inverter3x3}\PY{p}{(}\PY{n}{gammaDD}\PY{p}{)}
\end{Verbatim}

    Then, we will declare the gridfunctions related to the metric and build
the four metric using code from
\url{Tutorial-smallb2_Poynting_vector-Cartesian.ipynb}, which requires
the magnetic field \(B^i\). So, we will also define the magnetic field
as in \href{https://arxiv.org/pdf/1704.00599.pdf}{eq. 18}:
\[B^i = \epsilon^{ijk} \partial_j A_k,\] where \(\epsilon^{ijk}\) the
rank-3 Levi-Civita tensor, related to the rank-3 Levi-Civita symbol
\([ijk]\) and determinant of the three-metric \(\gamma\) by
\[\epsilon^{ijk} = [ijk]/\sqrt{\gamma}.\]

    \begin{Verbatim}[commandchars=\\\{\}]
{\color{incolor}In [{\color{incolor}2}]:} \PY{c+c1}{\PYZsh{} We already have a handy function to define the Levi\PYZhy{}Civita symbol in WeylScalars}
        \PY{k+kn}{import} \PY{n+nn}{WeylScal4NRPy.WeylScalars\PYZus{}Cartesian} \PY{k+kn}{as} \PY{n+nn}{weyl}
        \PY{n}{LeviCivitaDDD} \PY{o}{=} \PY{n}{weyl}\PY{o}{.}\PY{n}{define\PYZus{}LeviCivitaSymbol\PYZus{}rank3}\PY{p}{(}\PY{p}{)}
        \PY{n}{LeviCivitaUUU} \PY{o}{=} \PY{n}{ixp}\PY{o}{.}\PY{n}{zerorank3}\PY{p}{(}\PY{p}{)}
        \PY{k}{for} \PY{n}{i} \PY{o+ow}{in} \PY{n+nb}{range}\PY{p}{(}\PY{n}{DIM}\PY{p}{)}\PY{p}{:}
            \PY{k}{for} \PY{n}{j} \PY{o+ow}{in} \PY{n+nb}{range}\PY{p}{(}\PY{n}{DIM}\PY{p}{)}\PY{p}{:}
                \PY{k}{for} \PY{n}{k} \PY{o+ow}{in} \PY{n+nb}{range}\PY{p}{(}\PY{n}{DIM}\PY{p}{)}\PY{p}{:}
                    \PY{n}{LCijk} \PY{o}{=} \PY{n}{LeviCivitaDDD}\PY{p}{[}\PY{n}{i}\PY{p}{]}\PY{p}{[}\PY{n}{j}\PY{p}{]}\PY{p}{[}\PY{n}{k}\PY{p}{]}
                    \PY{n}{LeviCivitaDDD}\PY{p}{[}\PY{n}{i}\PY{p}{]}\PY{p}{[}\PY{n}{j}\PY{p}{]}\PY{p}{[}\PY{n}{k}\PY{p}{]} \PY{o}{=} \PY{n}{LCijk} \PY{o}{*} \PY{n}{sp}\PY{o}{.}\PY{n}{sqrt}\PY{p}{(}\PY{n}{gammadet}\PY{p}{)}
                    \PY{n}{LeviCivitaUUU}\PY{p}{[}\PY{n}{i}\PY{p}{]}\PY{p}{[}\PY{n}{j}\PY{p}{]}\PY{p}{[}\PY{n}{k}\PY{p}{]} \PY{o}{=} \PY{n}{LCijk} \PY{o}{/} \PY{n}{sp}\PY{o}{.}\PY{n}{sqrt}\PY{p}{(}\PY{n}{gammadet}\PY{p}{)}
        
        \PY{n}{AD\PYZus{}dD} \PY{o}{=} \PY{n}{ixp}\PY{o}{.}\PY{n}{declarerank2}\PY{p}{(}\PY{l+s+s2}{\PYZdq{}}\PY{l+s+s2}{AD\PYZus{}dD}\PY{l+s+s2}{\PYZdq{}}\PY{p}{,}\PY{l+s+s2}{\PYZdq{}}\PY{l+s+s2}{nosym}\PY{l+s+s2}{\PYZdq{}}\PY{p}{)}
        \PY{c+c1}{\PYZsh{} With Levi\PYZhy{}Civita and the derivative of A\PYZus{}i, we set B\PYZca{}i to the curl of A\PYZca{}i}
        \PY{n}{BU} \PY{o}{=} \PY{n}{ixp}\PY{o}{.}\PY{n}{zerorank1}\PY{p}{(}\PY{p}{)}
        \PY{k}{for} \PY{n}{i} \PY{o+ow}{in} \PY{n+nb}{range}\PY{p}{(}\PY{n}{DIM}\PY{p}{)}\PY{p}{:}
            \PY{k}{for} \PY{n}{j} \PY{o+ow}{in} \PY{n+nb}{range}\PY{p}{(}\PY{n}{DIM}\PY{p}{)}\PY{p}{:}
                \PY{k}{for} \PY{n}{k} \PY{o+ow}{in} \PY{n+nb}{range}\PY{p}{(}\PY{n}{DIM}\PY{p}{)}\PY{p}{:}
                    \PY{n}{BU}\PY{p}{[}\PY{n}{i}\PY{p}{]} \PY{o}{+}\PY{o}{=} \PY{n}{LeviCivitaUUU}\PY{p}{[}\PY{n}{i}\PY{p}{]}\PY{p}{[}\PY{n}{j}\PY{p}{]}\PY{p}{[}\PY{n}{k}\PY{p}{]} \PY{o}{*} \PY{n}{AD\PYZus{}dD}\PY{p}{[}\PY{n}{k}\PY{p}{]}\PY{p}{[}\PY{n}{j}\PY{p}{]}
        
        \PY{c+c1}{\PYZsh{} We use previous work to set u0 and smallb along with the four metric and its inverse}
        \PY{k+kn}{import} \PY{n+nn}{u0\PYZus{}smallb\PYZus{}Poynting\PYZus{}\PYZus{}Cartesian.u0\PYZus{}smallb\PYZus{}Poynting\PYZus{}\PYZus{}Cartesian} \PY{k+kn}{as} \PY{n+nn}{u0b}
        \PY{n}{u0b}\PY{o}{.}\PY{n}{compute\PYZus{}u0\PYZus{}smallb\PYZus{}Poynting\PYZus{}\PYZus{}Cartesian}\PY{p}{(}\PY{n}{gammaDD}\PY{p}{,}\PY{n}{betaU}\PY{p}{,}\PY{n}{alpha}\PY{p}{,}\PY{n}{ValenciavU}\PY{p}{,}\PY{n}{BU}\PY{p}{)}
\end{Verbatim}

    Recall that the four-metric \(g_{\mu\nu}\) is related to the
three-metric \(\gamma_{ij}\), index-lowered shift \(\beta_i\), and lapse
\(\alpha\) by\\
\[
g_{\mu\nu} = \begin{pmatrix} 
-\alpha^2 + \beta^k \beta_k & \beta_j \\
\beta_i & \gamma_{ij}
\end{pmatrix}.
\] This tensor and its inverse have already been built by the
u0\_smallb\_Poynting\_\_Cartesian.py module, so we can simply import the
variables.

    \begin{Verbatim}[commandchars=\\\{\}]
{\color{incolor}In [{\color{incolor}3}]:} \PY{n}{betaD} \PY{o}{=} \PY{n}{ixp}\PY{o}{.}\PY{n}{zerorank1}\PY{p}{(}\PY{p}{)}
        \PY{k}{for} \PY{n}{i} \PY{o+ow}{in} \PY{n+nb}{range}\PY{p}{(}\PY{n}{DIM}\PY{p}{)}\PY{p}{:}
            \PY{k}{for} \PY{n}{j} \PY{o+ow}{in} \PY{n+nb}{range}\PY{p}{(}\PY{n}{DIM}\PY{p}{)}\PY{p}{:}
                \PY{n}{betaD}\PY{p}{[}\PY{n}{i}\PY{p}{]} \PY{o}{=} \PY{n}{gammaDD}\PY{p}{[}\PY{n}{i}\PY{p}{]}\PY{p}{[}\PY{n}{j}\PY{p}{]} \PY{o}{*} \PY{n}{betaU}\PY{p}{[}\PY{n}{j}\PY{p}{]}
        
        \PY{c+c1}{\PYZsh{} We will now pull in the four metric and its inverse.}
        \PY{n}{g4DD} \PY{o}{=} \PY{n}{ixp}\PY{o}{.}\PY{n}{zerorank2}\PY{p}{(}\PY{n}{DIM}\PY{o}{=}\PY{l+m+mi}{4}\PY{p}{)}
        \PY{n}{g4UU} \PY{o}{=} \PY{n}{ixp}\PY{o}{.}\PY{n}{zerorank2}\PY{p}{(}\PY{n}{DIM}\PY{o}{=}\PY{l+m+mi}{4}\PY{p}{)}
        \PY{k}{for} \PY{n}{mu} \PY{o+ow}{in} \PY{n+nb}{range}\PY{p}{(}\PY{l+m+mi}{4}\PY{p}{)}\PY{p}{:}
            \PY{k}{for} \PY{n}{nu} \PY{o+ow}{in} \PY{n+nb}{range}\PY{p}{(}\PY{l+m+mi}{4}\PY{p}{)}\PY{p}{:}
                \PY{n}{g4DD}\PY{p}{[}\PY{n}{mu}\PY{p}{]}\PY{p}{[}\PY{n}{nu}\PY{p}{]} \PY{o}{=} \PY{n}{u0b}\PY{o}{.}\PY{n}{g4DD}\PY{p}{[}\PY{n}{mu}\PY{p}{]}\PY{p}{[}\PY{n}{nu}\PY{p}{]}
                \PY{n}{g4UU}\PY{p}{[}\PY{n}{mu}\PY{p}{]}\PY{p}{[}\PY{n}{nu}\PY{p}{]} \PY{o}{=} \PY{n}{u0b}\PY{o}{.}\PY{n}{g4UU}\PY{p}{[}\PY{n}{mu}\PY{p}{]}\PY{p}{[}\PY{n}{nu}\PY{p}{]}
\end{Verbatim}

    \subsection{Step 3: Build the spatial derivatives of the four
metric}\label{step-3-build-the-spatial-derivatives-of-the-four-metric}

\[\label{step3}\] {[}Back to Section \ref{top}{]}

We will also need spatial derivatives of the metric,
\(\partial_i g_{\mu\nu} = g_{\mu\nu,i}\). To do this, we will need
derivatives of the shift vector with its indexed lowered,
\(\beta_{i,j} = \partial_j \beta_i\). This becomes

\begin{align}
\beta_{i,j} &= \partial_j \beta_i \\
            &= \partial_j (\gamma_{ik} \beta^k) \\
            &= \gamma_{ik} \partial_j\beta^k + \beta^k \partial_j \gamma_{ik} \\
\beta_{i,j} &= \gamma_{ik} \beta^k_{\ ,j} + \beta^k \gamma_{ik,j} \\
\end{align}

In terms of the three-metric, lapse, and shift, we find \[
g_{\mu\nu,l} = \begin{pmatrix} 
-2\alpha \alpha_{,l} + \beta^k_{\ ,l} \beta_k + \beta^k \beta_{k,l} & \beta_{i,l} \\
\beta_{j,l} & \gamma_{ij,l}
\end{pmatrix}.
\]

Since this expression mixes Greek and Latin indices, we will need to
store the expressions for each of the three spatial derivatives as
separate variables. Also, consider the term
\(\beta_{i,j} = \partial_j \beta_i = \partial_j (\gamma_{ik} \beta^k) = \gamma_{ik} \partial_j\beta^k + \beta^k \partial_j \gamma_{ik} = \gamma_{ik} \beta^k_{\ ,j} + \beta^k \gamma_{ik,j}\),
that is,

\begin{align}
\beta_{i,j} &= \gamma_{ik} \beta^k_{\ ,j} + \beta^k \gamma_{ik,j}. \\
\end{align}

So, we will first set
\[ g_{00,l} = \underbrace{-2\alpha \alpha_{,l}}_{\rm Term\ 1} + \underbrace{\beta^k_{\ ,l} \beta_k}_{\rm Term\ 2} + \underbrace{\beta^k \beta_{k,l}}_{\rm Term\ 3} \]

    \begin{Verbatim}[commandchars=\\\{\}]
{\color{incolor}In [{\color{incolor}4}]:} \PY{c+c1}{\PYZsh{} Step 3: Build the spatial derivative of the four metric}
        \PY{c+c1}{\PYZsh{} Step 3a: Declare derivatives of grid functions. These will be handled by FD\PYZus{}outputC}
        \PY{n}{alpha\PYZus{}dD}   \PY{o}{=} \PY{n}{ixp}\PY{o}{.}\PY{n}{declarerank1}\PY{p}{(}\PY{l+s+s2}{\PYZdq{}}\PY{l+s+s2}{alpha\PYZus{}dD}\PY{l+s+s2}{\PYZdq{}}\PY{p}{)}
        \PY{n}{betaU\PYZus{}dD}   \PY{o}{=} \PY{n}{ixp}\PY{o}{.}\PY{n}{declarerank2}\PY{p}{(}\PY{l+s+s2}{\PYZdq{}}\PY{l+s+s2}{betaU\PYZus{}dD}\PY{l+s+s2}{\PYZdq{}}\PY{p}{,}\PY{l+s+s2}{\PYZdq{}}\PY{l+s+s2}{nosym}\PY{l+s+s2}{\PYZdq{}}\PY{p}{)}
        \PY{n}{gammaDD\PYZus{}dD} \PY{o}{=} \PY{n}{ixp}\PY{o}{.}\PY{n}{declarerank3}\PY{p}{(}\PY{l+s+s2}{\PYZdq{}}\PY{l+s+s2}{gammaDD\PYZus{}dD}\PY{l+s+s2}{\PYZdq{}}\PY{p}{,}\PY{l+s+s2}{\PYZdq{}}\PY{l+s+s2}{sym01}\PY{l+s+s2}{\PYZdq{}}\PY{p}{)}
        
        \PY{c+c1}{\PYZsh{} Step 3b: These derivatives will be constructed analytically.}
        \PY{n}{betaDdD}    \PY{o}{=} \PY{n}{ixp}\PY{o}{.}\PY{n}{zerorank2}\PY{p}{(}\PY{p}{)}
        
        \PY{n}{g4DDdD}     \PY{o}{=} \PY{n}{ixp}\PY{o}{.}\PY{n}{zerorank3}\PY{p}{(}\PY{n}{DIM}\PY{o}{=}\PY{l+m+mi}{4}\PY{p}{)}
        
        \PY{k}{for} \PY{n}{i} \PY{o+ow}{in} \PY{n+nb}{range}\PY{p}{(}\PY{n}{DIM}\PY{p}{)}\PY{p}{:}
            \PY{k}{for} \PY{n}{j} \PY{o+ow}{in} \PY{n+nb}{range}\PY{p}{(}\PY{n}{DIM}\PY{p}{)}\PY{p}{:}
                \PY{k}{for} \PY{n}{k} \PY{o+ow}{in} \PY{n+nb}{range}\PY{p}{(}\PY{n}{DIM}\PY{p}{)}\PY{p}{:}
                    \PY{c+c1}{\PYZsh{} \PYZbs{}gamma\PYZus{}\PYZob{}ik\PYZcb{} \PYZbs{}beta\PYZca{}k\PYZus{}\PYZob{},j\PYZcb{} + \PYZbs{}beta\PYZca{}k \PYZbs{}gamma\PYZus{}\PYZob{}ik,j\PYZcb{}}
                    \PY{n}{betaDdD}\PY{p}{[}\PY{n}{i}\PY{p}{]}\PY{p}{[}\PY{n}{j}\PY{p}{]} \PY{o}{=} \PY{n}{gammaDD}\PY{p}{[}\PY{n}{i}\PY{p}{]}\PY{p}{[}\PY{n}{k}\PY{p}{]} \PY{o}{*} \PY{n}{betaU\PYZus{}dD}\PY{p}{[}\PY{n}{k}\PY{p}{]}\PY{p}{[}\PY{n}{j}\PY{p}{]} \PY{o}{+} \PY{n}{betaU}\PY{p}{[}\PY{n}{k}\PY{p}{]} \PY{o}{*} \PY{n}{gammaDD\PYZus{}dD}\PY{p}{[}\PY{n}{i}\PY{p}{]}\PY{p}{[}\PY{n}{k}\PY{p}{]}\PY{p}{[}\PY{n}{j}\PY{p}{]}
        
        \PY{c+c1}{\PYZsh{} Step 3c: Set the 00 components}
        \PY{c+c1}{\PYZsh{} Step 3c.i: Term 1: \PYZhy{}2\PYZbs{}alpha \PYZbs{}alpha\PYZus{}\PYZob{},l\PYZcb{}}
        \PY{k}{for} \PY{n}{l} \PY{o+ow}{in} \PY{n+nb}{range}\PY{p}{(}\PY{n}{DIM}\PY{p}{)}\PY{p}{:}
            \PY{n}{g4DDdD}\PY{p}{[}\PY{l+m+mi}{0}\PY{p}{]}\PY{p}{[}\PY{l+m+mi}{0}\PY{p}{]}\PY{p}{[}\PY{n}{l}\PY{o}{+}\PY{l+m+mi}{1}\PY{p}{]} \PY{o}{=} \PY{o}{\PYZhy{}}\PY{l+m+mi}{2}\PY{o}{*}\PY{n}{alpha}\PY{o}{*}\PY{n}{alpha\PYZus{}dD}\PY{p}{[}\PY{n}{l}\PY{p}{]}
        
        \PY{c+c1}{\PYZsh{} Step 3c.ii: Term 2: \PYZbs{}beta\PYZca{}k\PYZus{}\PYZob{}\PYZbs{} ,l\PYZcb{} \PYZbs{}beta\PYZus{}k}
        \PY{k}{for} \PY{n}{l} \PY{o+ow}{in} \PY{n+nb}{range}\PY{p}{(}\PY{n}{DIM}\PY{p}{)}\PY{p}{:}
            \PY{k}{for} \PY{n}{k} \PY{o+ow}{in} \PY{n+nb}{range}\PY{p}{(}\PY{n}{DIM}\PY{p}{)}\PY{p}{:}
                \PY{n}{g4DDdD}\PY{p}{[}\PY{l+m+mi}{0}\PY{p}{]}\PY{p}{[}\PY{l+m+mi}{0}\PY{p}{]}\PY{p}{[}\PY{n}{l}\PY{o}{+}\PY{l+m+mi}{1}\PY{p}{]} \PY{o}{+}\PY{o}{=} \PY{n}{betaU\PYZus{}dD}\PY{p}{[}\PY{n}{k}\PY{p}{]}\PY{p}{[}\PY{n}{l}\PY{p}{]} \PY{o}{*} \PY{n}{betaD}\PY{p}{[}\PY{n}{k}\PY{p}{]}
                    
        \PY{c+c1}{\PYZsh{} Step 3c.iii: Term 3: \PYZbs{}beta\PYZca{}k \PYZbs{}beta\PYZus{}\PYZob{}k,l\PYZcb{}}
        \PY{k}{for} \PY{n}{l} \PY{o+ow}{in} \PY{n+nb}{range}\PY{p}{(}\PY{n}{DIM}\PY{p}{)}\PY{p}{:}
            \PY{k}{for} \PY{n}{k} \PY{o+ow}{in} \PY{n+nb}{range}\PY{p}{(}\PY{n}{DIM}\PY{p}{)}\PY{p}{:}
                \PY{n}{g4DDdD}\PY{p}{[}\PY{l+m+mi}{0}\PY{p}{]}\PY{p}{[}\PY{l+m+mi}{0}\PY{p}{]}\PY{p}{[}\PY{n}{l}\PY{o}{+}\PY{l+m+mi}{1}\PY{p}{]} \PY{o}{+}\PY{o}{=} \PY{n}{betaU}\PY{p}{[}\PY{n}{k}\PY{p}{]} \PY{o}{*} \PY{n}{betaDdD}\PY{p}{[}\PY{n}{k}\PY{p}{]}\PY{p}{[}\PY{n}{l}\PY{p}{]}
                    
\end{Verbatim}

    Now we will contruct the other components of \(g_{\mu\nu,l}\). We will
first construct \[ g_{i0,l} = g_{0i,l} = \beta_{i,l}, \] then
\[ g_{ij,l} = \gamma_{ij,l} \]

    \begin{Verbatim}[commandchars=\\\{\}]
{\color{incolor}In [{\color{incolor}5}]:} \PY{c+c1}{\PYZsh{} Step 3d: Set the i0 and 0j components}
        \PY{k}{for} \PY{n}{l} \PY{o+ow}{in} \PY{n+nb}{range}\PY{p}{(}\PY{n}{DIM}\PY{p}{)}\PY{p}{:}
            \PY{k}{for} \PY{n}{i} \PY{o+ow}{in} \PY{n+nb}{range}\PY{p}{(}\PY{n}{DIM}\PY{p}{)}\PY{p}{:}
                \PY{c+c1}{\PYZsh{} \PYZbs{}beta\PYZus{}\PYZob{}i,l\PYZcb{}}
                \PY{n}{g4DDdD}\PY{p}{[}\PY{n}{i}\PY{o}{+}\PY{l+m+mi}{1}\PY{p}{]}\PY{p}{[}\PY{l+m+mi}{0}\PY{p}{]}\PY{p}{[}\PY{n}{l}\PY{o}{+}\PY{l+m+mi}{1}\PY{p}{]} \PY{o}{=} \PY{n}{g4DDdD}\PY{p}{[}\PY{l+m+mi}{0}\PY{p}{]}\PY{p}{[}\PY{n}{i}\PY{o}{+}\PY{l+m+mi}{1}\PY{p}{]}\PY{p}{[}\PY{n}{l}\PY{o}{+}\PY{l+m+mi}{1}\PY{p}{]} \PY{o}{=} \PY{n}{betaDdD}\PY{p}{[}\PY{n}{i}\PY{p}{]}\PY{p}{[}\PY{n}{l}\PY{p}{]}
        
        \PY{c+c1}{\PYZsh{}Step 3e: Set the ij components}
        \PY{k}{for} \PY{n}{l} \PY{o+ow}{in} \PY{n+nb}{range}\PY{p}{(}\PY{n}{DIM}\PY{p}{)}\PY{p}{:}
            \PY{k}{for} \PY{n}{i} \PY{o+ow}{in} \PY{n+nb}{range}\PY{p}{(}\PY{n}{DIM}\PY{p}{)}\PY{p}{:}
                \PY{k}{for} \PY{n}{j} \PY{o+ow}{in} \PY{n+nb}{range}\PY{p}{(}\PY{n}{DIM}\PY{p}{)}\PY{p}{:}
                    \PY{c+c1}{\PYZsh{} \PYZbs{}gamma\PYZus{}\PYZob{}ij,l\PYZcb{}}
                    \PY{n}{g4DDdD}\PY{p}{[}\PY{n}{i}\PY{o}{+}\PY{l+m+mi}{1}\PY{p}{]}\PY{p}{[}\PY{n}{j}\PY{o}{+}\PY{l+m+mi}{1}\PY{p}{]}\PY{p}{[}\PY{n}{l}\PY{o}{+}\PY{l+m+mi}{1}\PY{p}{]} \PY{o}{=} \PY{n}{gammaDD\PYZus{}dD}\PY{p}{[}\PY{n}{i}\PY{p}{]}\PY{p}{[}\PY{n}{j}\PY{p}{]}\PY{p}{[}\PY{n}{l}\PY{p}{]}
\end{Verbatim}

    \section{\texorpdfstring{\(T^{\mu\nu}_{\rm EM}\) and its
derivatives}{T\^{}\{\textbackslash{}mu\textbackslash{}nu\}\_\{\textbackslash{}rm EM\} and its derivatives}}\label{tmunu_rm-em-and-its-derivatives}

\subsection{\texorpdfstring{Step 4: \(u^i\) and \(b^i\) and related
quantities}{Step 4: u\^{}i and b\^{}i and related quantities}}\label{step-4-ui-and-bi-and-related-quantities}

\[\label{step4}\] {[}Back to Section \ref{top}{]}

Now that the metric and its derivatives are out of the way, we will
return our attention to the electromagnetic stress-energy tensor, drawn
from eq. 27 of \href{https://arxiv.org/pdf/1310.3274.pdf}{this paper}:
\[T^{\mu \nu}_{\rm EM} = b^2 u^{\mu} u^{\nu} + \frac{b^2}{2} g^{\mu \nu} - b^{\mu} b^{\nu}.\]
We will need the four-velocity \(u^\mu\), which is related to the
Valencia 3-velocity \(v^i_{(n)}\) that we are using by\\

\begin{align}
u^i &= u^0 (\alpha v^i_{(n)} - \beta^i), \\
u_j &= \alpha u^0 \gamma_{ij} v^i_{(n)}, \\
\end{align}

and \(v^i_{(n)}\) is the Valencia three-velocity, as shown in
\href{https://arxiv.org/pdf/astro-ph/0503420.pdf}{Duez, et al, eqs. 53
and 56}. These have already been built by the
u0\_smallb\_Poynting\_\_Cartesian.py module, so we can simply import the
variables.

    \begin{Verbatim}[commandchars=\\\{\}]
{\color{incolor}In [{\color{incolor}6}]:} \PY{c+c1}{\PYZsh{} Step 4: \PYZdl{}u\PYZca{}i\PYZdl{} and \PYZdl{}b\PYZca{}i\PYZdl{} and related quantities}
        \PY{c+c1}{\PYZsh{} Step 4a: import the four\PYZhy{}velocity terms}
        \PY{c+c1}{\PYZsh{}u0 = par.Cparameters(\PYZdq{}REAL\PYZdq{},thismodule,\PYZdq{}u0\PYZdq{})}
        \PY{c+c1}{\PYZsh{}u0 = gri.register\PYZus{}gridfunctions(\PYZdq{}AUX\PYZdq{},\PYZdq{}u0\PYZdq{})}
        \PY{n}{uD} \PY{o}{=} \PY{n}{ixp}\PY{o}{.}\PY{n}{register\PYZus{}gridfunctions\PYZus{}for\PYZus{}single\PYZus{}rank1}\PY{p}{(}\PY{l+s+s2}{\PYZdq{}}\PY{l+s+s2}{AUX}\PY{l+s+s2}{\PYZdq{}}\PY{p}{,}\PY{l+s+s2}{\PYZdq{}}\PY{l+s+s2}{uD}\PY{l+s+s2}{\PYZdq{}}\PY{p}{)}
        \PY{n}{uU} \PY{o}{=} \PY{n}{ixp}\PY{o}{.}\PY{n}{register\PYZus{}gridfunctions\PYZus{}for\PYZus{}single\PYZus{}rank1}\PY{p}{(}\PY{l+s+s2}{\PYZdq{}}\PY{l+s+s2}{AUX}\PY{l+s+s2}{\PYZdq{}}\PY{p}{,}\PY{l+s+s2}{\PYZdq{}}\PY{l+s+s2}{uU}\PY{l+s+s2}{\PYZdq{}}\PY{p}{)}
        
        \PY{n}{u0} \PY{o}{=} \PY{n}{u0b}\PY{o}{.}\PY{n}{u0}
        \PY{k}{for} \PY{n}{i} \PY{o+ow}{in} \PY{n+nb}{range}\PY{p}{(}\PY{n}{DIM}\PY{p}{)}\PY{p}{:}
            \PY{n}{uD}\PY{p}{[}\PY{n}{i}\PY{p}{]} \PY{o}{=} \PY{n}{u0b}\PY{o}{.}\PY{n}{uD}\PY{p}{[}\PY{n}{i}\PY{p}{]}
            \PY{n}{uU}\PY{p}{[}\PY{n}{i}\PY{p}{]} \PY{o}{=} \PY{n}{u0b}\PY{o}{.}\PY{n}{uU}\PY{p}{[}\PY{n}{i}\PY{p}{]}
\end{Verbatim}

    We also need the vector \(b^{\mu}\) before we can compute this, which is
related to the magnetic field by

\begin{align}
b^0 &= \frac{1}{\sqrt{4\pi}} B^0_{\rm (u)} = \frac{u_j B^j}{\sqrt{4\pi}\alpha}, \\
b^i &= \frac{1}{\sqrt{4\pi}} B^i_{\rm (u)} = \frac{B^i + (u_j B^j) u^i}{\sqrt{4\pi}\alpha u^0}, \\
\end{align}

and

\begin{align}
B^i &= \frac{\tilde{B}^i}{\sqrt{\gamma}},
\end{align}

where \(B^i\) is the variable tracked by the HydroBase thorn in the
Einstein Toolkit. These have already been built by the
u0\_smallb\_Poynting\_\_Cartesian.py module, so we can simply import the
variables.

    \begin{Verbatim}[commandchars=\\\{\}]
{\color{incolor}In [{\color{incolor}7}]:} \PY{c+c1}{\PYZsh{} Step 4b: import the small b terms}
        \PY{n}{smallb4U} \PY{o}{=} \PY{n}{ixp}\PY{o}{.}\PY{n}{zerorank1}\PY{p}{(}\PY{n}{DIM}\PY{o}{=}\PY{l+m+mi}{4}\PY{p}{)}
        \PY{n}{smallb4D} \PY{o}{=} \PY{n}{ixp}\PY{o}{.}\PY{n}{zerorank1}\PY{p}{(}\PY{n}{DIM}\PY{o}{=}\PY{l+m+mi}{4}\PY{p}{)}
        \PY{k}{for} \PY{n}{mu} \PY{o+ow}{in} \PY{n+nb}{range}\PY{p}{(}\PY{l+m+mi}{4}\PY{p}{)}\PY{p}{:}
            \PY{n}{smallb4U}\PY{p}{[}\PY{n}{mu}\PY{p}{]} \PY{o}{=} \PY{n}{u0b}\PY{o}{.}\PY{n}{smallb4U}\PY{p}{[}\PY{n}{mu}\PY{p}{]}
            \PY{n}{smallb4D}\PY{p}{[}\PY{n}{mu}\PY{p}{]} \PY{o}{=} \PY{n}{u0b}\PY{o}{.}\PY{n}{smallb4D}\PY{p}{[}\PY{n}{mu}\PY{p}{]}
        
        \PY{n}{smallb2} \PY{o}{=} \PY{n}{u0b}\PY{o}{.}\PY{n}{smallb2}
\end{Verbatim}

     \#\# Step 5: Construct the electromagnetic stress-energy tensor
\[\label{step5}\]

{[}Back to Section \ref{top}{]}

We now have all the pieces to calculate the stress-energy tensor,
\[T^{\mu \nu}_{\rm EM} = \underbrace{b^2 u^{\mu} u^{\nu}}_{\rm Term\ 1} + 
\underbrace{\frac{b^2}{2} g^{\mu \nu}}_{\rm Term\ 2}
- \underbrace{b^{\mu} b^{\nu}}_{\rm Term\ 3}.\] Because \(u^0\) is a
separate variable, we could build the \(00\) component separately, then
the \(\mu0\) and \(0\nu\) components, and finally the \(\mu\nu\)
components. Alternatively, for clarity, we could create a temporary
variable \(u^\mu=\left( u^0, u^i \right)\)

    \begin{Verbatim}[commandchars=\\\{\}]
{\color{incolor}In [{\color{incolor}8}]:} \PY{c+c1}{\PYZsh{} Step 5: Construct the electromagnetic stress\PYZhy{}energy tensor}
        \PY{c+c1}{\PYZsh{} Step 5a: Set up the four\PYZhy{}velocity vector}
        \PY{n}{u4U} \PY{o}{=} \PY{n}{ixp}\PY{o}{.}\PY{n}{zerorank1}\PY{p}{(}\PY{n}{DIM}\PY{o}{=}\PY{l+m+mi}{4}\PY{p}{)}
        \PY{n}{u4U}\PY{p}{[}\PY{l+m+mi}{0}\PY{p}{]} \PY{o}{=} \PY{n}{u0}
        \PY{k}{for} \PY{n}{i} \PY{o+ow}{in} \PY{n+nb}{range}\PY{p}{(}\PY{n}{DIM}\PY{p}{)}\PY{p}{:}
            \PY{n}{u4U}\PY{p}{[}\PY{n}{i}\PY{o}{+}\PY{l+m+mi}{1}\PY{p}{]} \PY{o}{=} \PY{n}{uU}\PY{p}{[}\PY{n}{i}\PY{p}{]}
        
        \PY{c+c1}{\PYZsh{} Step 5b: Build T4EMUU itself}
        \PY{n}{T4EMUU} \PY{o}{=} \PY{n}{ixp}\PY{o}{.}\PY{n}{zerorank2}\PY{p}{(}\PY{n}{DIM}\PY{o}{=}\PY{l+m+mi}{4}\PY{p}{)}
        \PY{k}{for} \PY{n}{mu} \PY{o+ow}{in} \PY{n+nb}{range}\PY{p}{(}\PY{l+m+mi}{4}\PY{p}{)}\PY{p}{:}
            \PY{k}{for} \PY{n}{nu} \PY{o+ow}{in} \PY{n+nb}{range}\PY{p}{(}\PY{l+m+mi}{4}\PY{p}{)}\PY{p}{:}
                \PY{c+c1}{\PYZsh{} Term 1: \PYZbs{}underbrace\PYZob{}b\PYZca{}2 u\PYZca{}\PYZob{}\PYZbs{}mu\PYZcb{} u\PYZca{}\PYZob{}\PYZbs{}nu\PYZcb{}\PYZcb{}}
                \PY{n}{T4EMUU}\PY{p}{[}\PY{n}{mu}\PY{p}{]}\PY{p}{[}\PY{n}{nu}\PY{p}{]} \PY{o}{=} \PY{n}{smallb2}\PY{o}{*}\PY{n}{u4U}\PY{p}{[}\PY{n}{mu}\PY{p}{]}\PY{o}{*}\PY{n}{u4U}\PY{p}{[}\PY{n}{nu}\PY{p}{]}
        
        \PY{k}{for} \PY{n}{mu} \PY{o+ow}{in} \PY{n+nb}{range}\PY{p}{(}\PY{l+m+mi}{4}\PY{p}{)}\PY{p}{:}
            \PY{k}{for} \PY{n}{nu} \PY{o+ow}{in} \PY{n+nb}{range}\PY{p}{(}\PY{l+m+mi}{4}\PY{p}{)}\PY{p}{:}
                \PY{c+c1}{\PYZsh{} Term 2: b\PYZca{}2 / 2 g\PYZca{}\PYZob{}\PYZbs{}mu \PYZbs{}nu\PYZcb{}}
                \PY{n}{T4EMUU}\PY{p}{[}\PY{n}{mu}\PY{p}{]}\PY{p}{[}\PY{n}{nu}\PY{p}{]} \PY{o}{+}\PY{o}{=} \PY{n}{smallb2}\PY{o}{*}\PY{n}{g4UU}\PY{p}{[}\PY{n}{mu}\PY{p}{]}\PY{p}{[}\PY{n}{nu}\PY{p}{]}\PY{o}{/}\PY{l+m+mi}{2}
        
        \PY{k}{for} \PY{n}{mu} \PY{o+ow}{in} \PY{n+nb}{range}\PY{p}{(}\PY{l+m+mi}{4}\PY{p}{)}\PY{p}{:}
            \PY{k}{for} \PY{n}{nu} \PY{o+ow}{in} \PY{n+nb}{range}\PY{p}{(}\PY{l+m+mi}{4}\PY{p}{)}\PY{p}{:}
                \PY{c+c1}{\PYZsh{} Term 3: b\PYZca{}\PYZob{}\PYZbs{}mu\PYZcb{} b\PYZca{}\PYZob{}\PYZbs{}nu\PYZcb{}}
                \PY{n}{T4EMUU}\PY{p}{[}\PY{n}{mu}\PY{p}{]}\PY{p}{[}\PY{n}{nu}\PY{p}{]} \PY{o}{+}\PY{o}{=} \PY{o}{\PYZhy{}}\PY{n}{smallb4U}\PY{p}{[}\PY{n}{mu}\PY{p}{]}\PY{o}{*}\PY{n}{smallb4U}\PY{p}{[}\PY{n}{nu}\PY{p}{]}
\end{Verbatim}

    \subsection{Step 6: Derivatives of the electromagnetic stress-energy
tensor}\label{step-6-derivatives-of-the-electromagnetic-stress-energy-tensor}

\[\label{step6}\]

{[}Back to Section \ref{top}{]}

If we look at the evolution equation, we see that we will need spatial
derivatives of \(T^{\mu\nu}_{\rm EM}\). In previous tutorials, when
confronted with derivatives of complicated expressions, we have declared
those expressions as gridfunctions themselves, thus allowing NRPy+ to
take finite-difference derivatives of the expressions. This generally
reduces the error, because the alternative is to use a function of
several finite-difference derivatives, allowing more error to accumulate
than the extra gridfunction will introduce. While we will use that
technique for some of the subexpressions of \(T^{\mu\nu}_{\rm EM}\), we
don't want to rely on it for the whole expression; doing so would
require us to take the derivative of the magnetic field \(B^i\), which
is itself found by finite-differencing the vector potential \(A_i\). It
requires some finesse, then, to find \(B^i\) inside the ghost zones, and
those values of \(B^i\) are necessarily less accurate; taking another
derivative from them would only compound the problem. Instead, we will
take analytic derivatives of \(T^{\mu\nu}_{\rm EM}\).

We will now now take these spatial derivatives of
\(T^{\mu\nu}_{\rm EM}\), applying the chain rule until it is only in
terms of basic gridfunctions and their derivatives: \(\alpha\),
\(\beta^i\), \(\gamma_{ij}\), \(A_i\), and the four-velocity \(u^i\).
Along the way, we will also set up useful temporary variables
representing the steps of chain rule: * \(B^i\) (already computed in
terms of \(A_k\)), * \(B^i_{,l}\), * \(b^i\) and \(b_i\) (already
computed), * \(b^i_{,k}\), * \(b^2\) (already computed), * and
\(\left(b^2\right)_{,j}\).

(The variables not already computed will not be seen by the ETK; they
simply help to organize the NRPy+ code.)

We will need the definitions of \(\tilde{B}^i\) and \(B^i\) in terms of
\(B^i\) and \(A_i\):

\begin{align}
\tilde{B}^i &= \sqrt{\gamma} B^i \\
B^i &= \epsilon^{ijk} \partial_j A_k \\
\end{align}

So then,

\begin{align}
\partial_j T^{j}_{{\rm EM} i} &= \partial_j (\gamma_{ki} T^{kj}_{\rm EM}) \\
&= \partial_j [\gamma_{ki} (b^2 u^j u^k + \frac{b^2}{2} g^{jk} - b^j b^k)] \\
&= \underbrace{\gamma_{ki,j} T^{kj}_{\rm EM}}_{\rm Term\ 1} + \gamma_{ki} \left( \underbrace{\partial_j \left(b^2 u^j u^k\right)}_{\rm Term\ 2} + \underbrace{\partial_j \left(\frac{b^2}{2} g^{jk}\right)}_{\rm Term\ 3} - \underbrace{\partial_j \left(b^j b^k\right)}_{\rm Term\ 4} \right) \\
\end{align}

Following the product and chain rules for each term, we find that

\begin{align}
{\rm Term\ 2} &= \partial_j (b^2 u^j u^k) \\
              &= \partial_j b^2 u^j u^k + b^2 \partial_j u^j u^k + b^2 u^j \partial_j u_k \\
              &= \left(b^2\right)_{,j} u^j u^k + b^2 u^j_{,j} u^k + b^2 u^j u^k_{,j} \\
{\rm Term\ 3} &= \partial_j \left(\frac{b^2}{2} g^{jk}\right) \\
              &= \frac{1}{2} \left( \partial_j b^2 g^{jk} + b^2 \partial_j g^{jk} \right) \\
              &= \frac{1}{2} \left(b^2\right)_{,j} g^{jk} + \frac{b^2}{2} g^{jk}_{\ ,j} \\
{\rm Term\ 4} &= \partial_j (b^j b^k) \\
              &= b^j_{,j} b^k + b^j b^k_{,j}\\
\end{align}

So,

\begin{align}
\partial_j T^{j}_{{\rm EM} i} &= \gamma_{ki,j} T^{kj}_{\rm EM} \\
&+ \gamma_{ki} \left(\left(b^2\right)_{,j} u^j u^k +b^2 u^j_{,j} u^k + b^2 u^j u^k_{,j} + \frac{1}{2}\left(b^2\right)_{,j} g^{jk} + \frac{b^2}{2} g^{jk}_{\ ,j} + b^j_{,j} b^k + b^j b^k_{,j}\right).
\end{align}

 \textbf{List of Derivatives} Note that this is in terms of the
derivatives of several other quantities: * \(B^i_{,l}\):
Section \ref{bideriv} * \(b^i_{,k}\): Section \ref{bideriv} * The
derivative of \(b^2 = g_{\mu\nu} b^\mu b^\nu\), Section \ref{b2deriv}. *
Putting it together: Section \ref{alltogether}

\begin{align}
%u^i_{,j} &= u^0_{,j} (\alpha v^i_{(n)} - \beta^i) + u^0 (\alpha_{,j} v^i_{(n)} + \alpha v^i_{(n),j} - \beta^i_{,j}) \\
%u_{j,k} &= \alpha_{,k} u^0 \gamma_{ij} v^i_{(n)} + \alpha u^0_{,k} \gamma_{ij} v^i_{(n)} + \alpha u^0 \gamma_{ij,k} v^i_{(n)} + \alpha u^0 \gamma_{ij} v^i_{(n),k} \\
%b^i_{,k} &= \frac{1}{\sqrt{4 \pi}} \frac{\left(\alpha u^0\right)  \left(B^i_{,k} + u_{j,k} B^j u^i + u_j B^j_{,k} u^i + u_j B^j u^i_{,k}\right) - \left(B^i + (u_j B^j) u^i\right) \partial_k \left(\alpha u^0\right)}{\left(\alpha u^0\right)^2} \\
%B^i_{,l} &= -\frac{\gamma_{,l}}{2\gamma} B^i + \epsilon^{ijk} A_{k,jl}. \\
\end{align}

     {[}Back to the Section \ref{table2}{]}

First, we will build the derivatives of the magnetic field. Since
\(b^i\) is a function of \(B^i\), we will start from the definition of
\(B^i\) in terms of \(A_i\),
\(B^i = \frac{[ijk]}{\sqrt{\gamma}} \partial_j A_k\). We will first
apply the product rule, noting that the symbol \([ijk]\) consists purely
of the integers \(-1, 0, 1\) and thus can be treated as a constant in
this process.

\begin{align}
B^i_{,l} &= \partial_l \left( \frac{[ijk]}{\sqrt{\gamma}} \partial_j A_k \right)  \\
         &= [ijk] \partial_l \left( \frac{1}{\sqrt{\gamma}}\right) \partial_j A_k + \frac{[ijk]}{\sqrt{\gamma}} \partial_l \partial_j A_k \\
         &= [ijk]\left(-\frac{\gamma_{,l}}{2\gamma^{3/2}}\right) \partial_j A_k + \frac{[ijk]}{\sqrt{\gamma}} \partial_l \partial_j A_k \\
\end{align}

Now, we will substitute back in for the definition of the Levi-Civita
tensor: \(\epsilon^{ijk} = [ijk] / \sqrt{\gamma}\). Then will substitute
the magnetic field \(B^i\) back in.

\begin{align}
B^i_{,l} &= -\frac{\gamma_{,l}}{2\gamma} \epsilon^{ijk} \partial_j A_k + \epsilon^{ijk} \partial_l \partial_j A_k \\
         &= -\frac{\gamma_{,l}}{2\gamma} B^i + \epsilon^{ijk} A_{k,jl}, \\
\end{align}

Thus, the expression we are left with for the derivatives of the
magnetic field is:

\begin{align}
B^i_{,l} &= \underbrace{-\frac{\gamma_{,l}}{2\gamma} B^i}_{\rm Term\ 1} + \underbrace{\epsilon^{ijk} A_{k,jl}}_{\rm Term\ 2}, \\
\end{align}

where \(\epsilon^{ijk} = [ijk] / \sqrt{\gamma}\) is the antisymmetric
Levi-Civita tensor and \(\gamma\) is the determinant of the
three-metric.

    \begin{Verbatim}[commandchars=\\\{\}]
{\color{incolor}In [{\color{incolor}9}]:} \PY{c+c1}{\PYZsh{} Step 6: Derivatives of the electromagnetic stress\PYZhy{}energy tensor}
        \PY{c+c1}{\PYZsh{} Step 6a: Declare gridfunctions and their derivatives that will be useful for TEMUD\PYZus{}dD\PYZus{}contracted (as needed)}
        
        \PY{c+c1}{\PYZsh{} Step 6b: Construct the derivatives of the magnetic field.}
        \PY{n}{gammadet\PYZus{}dD} \PY{o}{=} \PY{n}{ixp}\PY{o}{.}\PY{n}{declarerank1}\PY{p}{(}\PY{l+s+s2}{\PYZdq{}}\PY{l+s+s2}{gammadet\PYZus{}dD}\PY{l+s+s2}{\PYZdq{}}\PY{p}{)}
            
        \PY{n}{AD\PYZus{}dDD} \PY{o}{=} \PY{n}{ixp}\PY{o}{.}\PY{n}{declarerank3}\PY{p}{(}\PY{l+s+s2}{\PYZdq{}}\PY{l+s+s2}{AD\PYZus{}dDD}\PY{l+s+s2}{\PYZdq{}}\PY{p}{,}\PY{l+s+s2}{\PYZdq{}}\PY{l+s+s2}{sym12}\PY{l+s+s2}{\PYZdq{}}\PY{p}{)}           
        \PY{c+c1}{\PYZsh{} The other partial derivatives of B\PYZca{}i}
        \PY{n}{BU\PYZus{}dD} \PY{o}{=} \PY{n}{ixp}\PY{o}{.}\PY{n}{zerorank2}\PY{p}{(}\PY{p}{)}
        \PY{k}{for} \PY{n}{i} \PY{o+ow}{in} \PY{n+nb}{range}\PY{p}{(}\PY{n}{DIM}\PY{p}{)}\PY{p}{:}
            \PY{k}{for} \PY{n}{l} \PY{o+ow}{in} \PY{n+nb}{range}\PY{p}{(}\PY{n}{DIM}\PY{p}{)}\PY{p}{:}
                \PY{c+c1}{\PYZsh{} Term 1: \PYZhy{}\PYZbs{}gamma\PYZus{}\PYZob{},l\PYZcb{} / (2\PYZbs{}gamma) B\PYZca{}i}
                \PY{n}{BU\PYZus{}dD}\PY{p}{[}\PY{n}{i}\PY{p}{]}\PY{p}{[}\PY{n}{l}\PY{p}{]} \PY{o}{=} \PY{o}{\PYZhy{}}\PY{n}{gammadet\PYZus{}dD}\PY{p}{[}\PY{n}{l}\PY{p}{]}\PY{o}{*}\PY{n}{BU}\PY{p}{[}\PY{n}{i}\PY{p}{]}\PY{o}{/}\PY{p}{(}\PY{l+m+mi}{2}\PY{o}{*}\PY{n}{gammadet}\PY{p}{)}
                
        \PY{k}{for} \PY{n}{i} \PY{o+ow}{in} \PY{n+nb}{range}\PY{p}{(}\PY{n}{DIM}\PY{p}{)}\PY{p}{:}
            \PY{k}{for} \PY{n}{l} \PY{o+ow}{in} \PY{n+nb}{range}\PY{p}{(}\PY{n}{DIM}\PY{p}{)}\PY{p}{:}
                \PY{k}{for} \PY{n}{j} \PY{o+ow}{in} \PY{n+nb}{range}\PY{p}{(}\PY{n}{DIM}\PY{p}{)}\PY{p}{:}
                    \PY{k}{for} \PY{n}{k} \PY{o+ow}{in} \PY{n+nb}{range}\PY{p}{(}\PY{n}{DIM}\PY{p}{)}\PY{p}{:}
                        \PY{c+c1}{\PYZsh{} Term 2: \PYZbs{}epsilon\PYZca{}\PYZob{}ijk\PYZcb{} A\PYZus{}\PYZob{}k,jl\PYZcb{}}
                        \PY{n}{BU\PYZus{}dD}\PY{p}{[}\PY{n}{i}\PY{p}{]}\PY{p}{[}\PY{n}{l}\PY{p}{]} \PY{o}{+}\PY{o}{=} \PY{n}{LeviCivitaUUU}\PY{p}{[}\PY{n}{i}\PY{p}{]}\PY{p}{[}\PY{n}{j}\PY{p}{]}\PY{p}{[}\PY{n}{k}\PY{p}{]} \PY{o}{*} \PY{n}{AD\PYZus{}dDD}\PY{p}{[}\PY{n}{k}\PY{p}{]}\PY{p}{[}\PY{n}{j}\PY{p}{]}\PY{p}{[}\PY{n}{l}\PY{p}{]}
\end{Verbatim}

    Now, we will code the derivatives of the spatial components of
\(b^{\mu}\), \(b^i\): \[
b^i_{,k} = \frac{1}{\sqrt{4 \pi}} \frac{\left(\alpha u^0\right)  \left(B^i_{,k} + u_{j,k} B^j u^i + u_j B^j_{,k} u^i + u_j B^j u^i_{,k}\right) - \left(B^i + (u_j B^j) u^i\right) \partial_k \left(\alpha u^0\right)}{\left(\alpha u^0\right)^2}. 
\]

We should note that while \(b^\mu\) is a four-vector (and the code
reflects this: \(\text{smallb4U}\) and \(\text{smallb4U}\) have
\(\text{DIM=4}\)), we only need the spatial components. Since
\(b^0_{,k}\) would require a separate derivation, we will simply let the
vector be three-dimensional.

 {[}Back to the Section \ref{table2}{]}

Let's go into a little more detail on where this comes from. We start
from the definition
\[b^i = \frac{B^i + (u_j B^j) u^i}{\sqrt{4\pi}\alpha u^0};\] We then
apply the quotient rule:

\begin{align}
b^i_{,k} &= \frac{\left(\sqrt{4\pi}\alpha u^0\right) \partial_k \left(B^i + (u_j B^j) u^i\right) - \left(B^i + (u_j B^j) u^i\right) \partial_k \left(\sqrt{4\pi}\alpha u^0\right)}{\left(\sqrt{4\pi}\alpha u^0\right)^2} \\
&= \frac{1}{\sqrt{4 \pi}} \frac{\left(\alpha u^0\right) \partial_k \left(B^i + (u_j B^j) u^i\right) - \left(B^i + (u_j B^j) u^i\right) \partial_k \left(\alpha u^0\right)}{\left(\alpha u^0\right)^2} \\
\end{align}

Note that \(\left( \alpha u^0 \right)\) is being used as its own
gridfunction, so we can be done with that term. We will now apply the
product rule to the term
\(\partial_k \left(B^i + (u_j B^j) u^i\right) = B^i_{,k} + u_{j,k} B^j u^i + u_j B^j_{,k} u^i + u_j B^j u^i_{,k}\).
So,
\[ b^i_{,k} = \frac{1}{\sqrt{4 \pi}} \frac{\left(\alpha u^0\right)  \left(B^i_{,k} + u_{j,k} B^j u^i + u_j B^j_{,k} u^i + u_j B^j u^i_{,k}\right) - \left(B^i + (u_j B^j) u^i\right) \partial_k \left(\alpha u^0\right)}{\left(\alpha u^0\right)^2}. \]

It will be easier to code this up if we rearrange these terms to group
together the terms that involve contractions over \(j\). Doing that, we
find \[
b^i_{,k} = \frac{\overbrace{\alpha u^0 B^i_{,k} - B^i \partial_k (\alpha u^0)}^{\rm Term\ Num1} + \overbrace{\left( \alpha u^0 \right) \left( u_{j,k} B^j u^i + u_j B^j_{,k} u^i + u_j B^j u^i_{,k} \right)}^{\rm Term\ Num2.a} - \overbrace{\left( u_j B^j u^i \right) \partial_k \left( \alpha u^0 \right) }^{\rm Term\ Num2.b}}{\underbrace{\sqrt{4 \pi} \left( \alpha u^0 \right)^2}_{\rm Term\ Denom}}.
\]

    \begin{Verbatim}[commandchars=\\\{\}]
{\color{incolor}In [{\color{incolor}10}]:} \PY{n}{alphau0} \PY{o}{=} \PY{n}{gri}\PY{o}{.}\PY{n}{register\PYZus{}gridfunctions}\PY{p}{(}\PY{l+s+s2}{\PYZdq{}}\PY{l+s+s2}{AUX}\PY{l+s+s2}{\PYZdq{}}\PY{p}{,}\PY{l+s+s2}{\PYZdq{}}\PY{l+s+s2}{alphau0}\PY{l+s+s2}{\PYZdq{}}\PY{p}{)}
         \PY{n}{alphau0} \PY{o}{=} \PY{n}{alpha} \PY{o}{*} \PY{n}{u0}
         \PY{n}{alphau0\PYZus{}dD} \PY{o}{=} \PY{n}{ixp}\PY{o}{.}\PY{n}{declarerank1}\PY{p}{(}\PY{l+s+s2}{\PYZdq{}}\PY{l+s+s2}{alphau0\PYZus{}dD}\PY{l+s+s2}{\PYZdq{}}\PY{p}{)}
         \PY{n}{uU\PYZus{}dD} \PY{o}{=} \PY{n}{ixp}\PY{o}{.}\PY{n}{declarerank2}\PY{p}{(}\PY{l+s+s2}{\PYZdq{}}\PY{l+s+s2}{uU\PYZus{}dD}\PY{l+s+s2}{\PYZdq{}}\PY{p}{,}\PY{l+s+s2}{\PYZdq{}}\PY{l+s+s2}{nosym}\PY{l+s+s2}{\PYZdq{}}\PY{p}{)}
         \PY{n}{uD\PYZus{}dD} \PY{o}{=} \PY{n}{ixp}\PY{o}{.}\PY{n}{declarerank2}\PY{p}{(}\PY{l+s+s2}{\PYZdq{}}\PY{l+s+s2}{uD\PYZus{}dD}\PY{l+s+s2}{\PYZdq{}}\PY{p}{,}\PY{l+s+s2}{\PYZdq{}}\PY{l+s+s2}{nosym}\PY{l+s+s2}{\PYZdq{}}\PY{p}{)}
         
         \PY{c+c1}{\PYZsh{} Step 6c: Construct derivatives of the small b vector}
         \PY{c+c1}{\PYZsh{} smallbU\PYZus{}dD represents the derivative of smallb4U}
         \PY{n}{smallbU\PYZus{}dD} \PY{o}{=} \PY{n}{ixp}\PY{o}{.}\PY{n}{zerorank2}\PY{p}{(}\PY{p}{)}
         \PY{k}{for} \PY{n}{i} \PY{o+ow}{in} \PY{n+nb}{range}\PY{p}{(}\PY{n}{DIM}\PY{p}{)}\PY{p}{:}
             \PY{k}{for} \PY{n}{k} \PY{o+ow}{in} \PY{n+nb}{range}\PY{p}{(}\PY{n}{DIM}\PY{p}{)}\PY{p}{:}
                 \PY{c+c1}{\PYZsh{} Term Num1: \PYZbs{}alpha u\PYZca{}0 B\PYZca{}i\PYZus{}\PYZob{},k\PYZcb{} \PYZhy{} B\PYZca{}i \PYZbs{}partial\PYZus{}k (\PYZbs{}alpha u\PYZca{}0)}
                 \PY{n}{smallbU\PYZus{}dD}\PY{p}{[}\PY{n}{i}\PY{p}{]}\PY{p}{[}\PY{n}{k}\PY{p}{]} \PY{o}{+}\PY{o}{=} \PY{n}{alphau0}\PY{o}{*}\PY{n}{BU\PYZus{}dD}\PY{p}{[}\PY{n}{i}\PY{p}{]}\PY{p}{[}\PY{n}{k}\PY{p}{]}\PY{o}{\PYZhy{}}\PY{n}{BU}\PY{p}{[}\PY{n}{i}\PY{p}{]}\PY{o}{*}\PY{n}{alphau0\PYZus{}dD}\PY{p}{[}\PY{n}{k}\PY{p}{]}
         
         \PY{k}{for} \PY{n}{i} \PY{o+ow}{in} \PY{n+nb}{range}\PY{p}{(}\PY{n}{DIM}\PY{p}{)}\PY{p}{:}
             \PY{k}{for} \PY{n}{k} \PY{o+ow}{in} \PY{n+nb}{range}\PY{p}{(}\PY{n}{DIM}\PY{p}{)}\PY{p}{:}
                 \PY{k}{for} \PY{n}{j} \PY{o+ow}{in} \PY{n+nb}{range}\PY{p}{(}\PY{n}{DIM}\PY{p}{)}\PY{p}{:}
                     \PY{c+c1}{\PYZsh{} Term Num2.a: terms that require contractions over k, and thus an extra loop.}
                     \PY{c+c1}{\PYZsh{} ( \PYZbs{}alpha u\PYZca{}0 ) (  u\PYZus{}\PYZob{}j,k\PYZcb{} B\PYZca{}j u\PYZca{}i }
                     \PY{c+c1}{\PYZsh{}                 + u\PYZus{}j B\PYZca{}j\PYZus{}\PYZob{},k\PYZcb{} u\PYZca{}i }
                     \PY{c+c1}{\PYZsh{}                 + u\PYZus{}j B\PYZca{}j u\PYZca{}i\PYZus{}\PYZob{},k\PYZcb{} )}
                     \PY{n}{smallbU\PYZus{}dD}\PY{p}{[}\PY{n}{i}\PY{p}{]}\PY{p}{[}\PY{n}{k}\PY{p}{]} \PY{o}{+}\PY{o}{=} \PY{n}{alphau0}\PY{o}{*}\PY{p}{(}\PY{n}{uD\PYZus{}dD}\PY{p}{[}\PY{n}{j}\PY{p}{]}\PY{p}{[}\PY{n}{k}\PY{p}{]}\PY{o}{*}\PY{n}{BU}\PY{p}{[}\PY{n}{j}\PY{p}{]}\PY{o}{*}\PY{n}{uU}\PY{p}{[}\PY{n}{i}\PY{p}{]}\PYZbs{}
                                                  \PY{o}{+}\PY{n}{uD}\PY{p}{[}\PY{n}{j}\PY{p}{]}\PY{o}{*}\PY{n}{BU\PYZus{}dD}\PY{p}{[}\PY{n}{j}\PY{p}{]}\PY{p}{[}\PY{n}{k}\PY{p}{]}\PY{o}{*}\PY{n}{uU}\PY{p}{[}\PY{n}{i}\PY{p}{]}\PYZbs{}
                                                  \PY{o}{+}\PY{n}{uD}\PY{p}{[}\PY{n}{j}\PY{p}{]}\PY{o}{*}\PY{n}{BU}\PY{p}{[}\PY{n}{j}\PY{p}{]}\PY{o}{*}\PY{n}{uU\PYZus{}dD}\PY{p}{[}\PY{n}{i}\PY{p}{]}\PY{p}{[}\PY{n}{k}\PY{p}{]}\PY{p}{)}
         
         \PY{k}{for} \PY{n}{i} \PY{o+ow}{in} \PY{n+nb}{range}\PY{p}{(}\PY{n}{DIM}\PY{p}{)}\PY{p}{:}
             \PY{k}{for} \PY{n}{k} \PY{o+ow}{in} \PY{n+nb}{range}\PY{p}{(}\PY{n}{DIM}\PY{p}{)}\PY{p}{:}
                 \PY{k}{for} \PY{n}{j} \PY{o+ow}{in} \PY{n+nb}{range}\PY{p}{(}\PY{n}{DIM}\PY{p}{)}\PY{p}{:}
                     \PY{c+c1}{\PYZsh{}Term 2.b (More contractions over k): ( u\PYZus{}j B\PYZca{}j u\PYZca{}i ) \PYZbs{}partial\PYZus{}k ( \PYZbs{}alpha u\PYZca{}0 )}
                     \PY{n}{smallbU\PYZus{}dD}\PY{p}{[}\PY{n}{i}\PY{p}{]}\PY{p}{[}\PY{n}{k}\PY{p}{]} \PY{o}{+}\PY{o}{=} \PY{o}{\PYZhy{}}\PY{p}{(}\PY{n}{uD}\PY{p}{[}\PY{n}{j}\PY{p}{]}\PY{o}{*}\PY{n}{BU}\PY{p}{[}\PY{n}{j}\PY{p}{]}\PY{o}{*}\PY{n}{uU}\PY{p}{[}\PY{n}{i}\PY{p}{]}\PY{p}{)}\PY{o}{*}\PY{n}{alphau0\PYZus{}dD}\PY{p}{[}\PY{n}{k}\PY{p}{]}
                     
         \PY{k}{for} \PY{n}{i} \PY{o+ow}{in} \PY{n+nb}{range}\PY{p}{(}\PY{n}{DIM}\PY{p}{)}\PY{p}{:}
             \PY{k}{for} \PY{n}{k} \PY{o+ow}{in} \PY{n+nb}{range}\PY{p}{(}\PY{n}{DIM}\PY{p}{)}\PY{p}{:}
                 \PY{c+c1}{\PYZsh{} Term Denom requires us to divide the whole expressions through by sqrt(4 pi) * (alpha u\PYZca{}0)\PYZca{}2}
                 \PY{n}{smallbU\PYZus{}dD}\PY{p}{[}\PY{n}{i}\PY{p}{]}\PY{p}{[}\PY{n}{k}\PY{p}{]} \PY{o}{/}\PY{o}{=} \PY{n}{sp}\PY{o}{.}\PY{n}{sqrt}\PY{p}{(}\PY{l+m+mi}{4}\PY{o}{*}\PY{n}{M\PYZus{}PI}\PY{p}{)} \PY{o}{*} \PY{n}{alphau0} \PY{o}{*} \PY{n}{alphau0}
\end{Verbatim}

     {[}Back to the Section \ref{table2}{]}

Here, we will take the derivative of \(b^2 = g_{\mu\nu} b^\mu b^\nu\).
Using the product rule,

\begin{align}
\left(b^2\right)_{,j} &= \partial_j \left( g_{\mu\nu} b^\mu b^\nu \right) \\
                      &= g_{\mu\nu,j} b^\mu b^\nu + g_{\mu\nu} b^\mu_{,j} b^\nu + g_{\mu\nu} b^\mu b^\nu_{,j} \\
                      &= g_{\mu\nu,j} b^\mu b^\nu + 2 g_{\mu\nu} b^\mu_{,j} b^\nu.
\end{align}

We have already defined the derivatives of the four-metric
\(g_{\mu\nu,j}\); we have also defined the derivatives of spatial
components of \(b^\mu\), \(b^i_{,k}\). Now, however, we will need the
take derivatives of the temporal component: \(b^0_{,j}\). Starting with
the definition, and using the quotient rule:

\begin{align}
b^0 &= \frac{u_k B^k}{\sqrt{4\pi}\alpha}, \\
\rightarrow b^0_{,j} &= \frac{1}{\sqrt{4\pi}} \frac{\alpha \left( u_{k,j} B^k + u_k B^k_{,j} \right) - u_k B^k \alpha_{,j}}{\alpha^2} \\
    &= \frac{\alpha u_{k,j} B^k + \alpha u_k B^k_{,j} - \alpha_{,j} u_k B^k}{\sqrt{4\pi} \alpha^2}.
\end{align}

We will first code the numerator, and then divide through by the
denominator.

    \begin{Verbatim}[commandchars=\\\{\}]
{\color{incolor}In [{\color{incolor}11}]:} \PY{c+c1}{\PYZsh{} First construct the derivative b\PYZca{}0\PYZus{}\PYZob{},j\PYZcb{}}
         \PY{c+c1}{\PYZsh{} This four\PYZhy{}vector will make b\PYZca{}2 simpler:}
         \PY{n}{smallb4U\PYZus{}dD} \PY{o}{=} \PY{n}{ixp}\PY{o}{.}\PY{n}{zerorank2}\PY{p}{(}\PY{n}{DIM}\PY{o}{=}\PY{l+m+mi}{4}\PY{p}{)}
         \PY{c+c1}{\PYZsh{} Fill in the zeroth component}
         \PY{k}{for} \PY{n}{j} \PY{o+ow}{in} \PY{n+nb}{range}\PY{p}{(}\PY{n}{DIM}\PY{p}{)}\PY{p}{:}
             \PY{k}{for} \PY{n}{k} \PY{o+ow}{in} \PY{n+nb}{range}\PY{p}{(}\PY{n}{DIM}\PY{p}{)}\PY{p}{:}
                 \PY{c+c1}{\PYZsh{} The numerator:  \PYZbs{}alpha u\PYZus{}\PYZob{}k,j\PYZcb{} B\PYZca{}k }
                 \PY{c+c1}{\PYZsh{}               + \PYZbs{}alpha u\PYZus{}k B\PYZca{}k\PYZus{}\PYZob{},j\PYZcb{} }
                 \PY{c+c1}{\PYZsh{}               \PYZhy{} \PYZbs{}alpha\PYZus{}\PYZob{},j\PYZcb{} u\PYZus{}k B\PYZca{}k}
                 \PY{n}{smallb4U\PYZus{}dD}\PY{p}{[}\PY{l+m+mi}{0}\PY{p}{]}\PY{p}{[}\PY{n}{j}\PY{o}{+}\PY{l+m+mi}{1}\PY{p}{]} \PY{o}{=}   \PY{n}{alpha}\PY{o}{*}\PY{n}{uD\PYZus{}dD}\PY{p}{[}\PY{n}{k}\PY{p}{]}\PY{p}{[}\PY{n}{j}\PY{p}{]}\PY{o}{*}\PY{n}{BU}\PY{p}{[}\PY{n}{k}\PY{p}{]} \PYZbs{}
                                       \PY{o}{+} \PY{n}{alpha}\PY{o}{*}\PY{n}{uD}\PY{p}{[}\PY{n}{k}\PY{p}{]}\PY{o}{*}\PY{n}{BU\PYZus{}dD}\PY{p}{[}\PY{n}{k}\PY{p}{]}\PY{p}{[}\PY{n}{j}\PY{p}{]} \PYZbs{}
                                       \PY{o}{\PYZhy{}} \PY{n}{alpha\PYZus{}dD}\PY{p}{[}\PY{n}{j}\PY{p}{]}\PY{o}{*}\PY{n}{uD}\PY{p}{[}\PY{n}{k}\PY{p}{]}\PY{o}{*}\PY{n}{BU}\PY{p}{[}\PY{n}{k}\PY{p}{]}
         \PY{k}{for} \PY{n}{j} \PY{o+ow}{in} \PY{n+nb}{range}\PY{p}{(}\PY{n}{DIM}\PY{p}{)}\PY{p}{:}
             \PY{c+c1}{\PYZsh{} Divide through by the denominator: \PYZbs{}sqrt\PYZob{}4\PYZbs{}pi\PYZcb{} \PYZbs{}alpha\PYZca{}2}
             \PY{n}{smallb4U\PYZus{}dD}\PY{p}{[}\PY{l+m+mi}{0}\PY{p}{]}\PY{p}{[}\PY{n}{j}\PY{o}{+}\PY{l+m+mi}{1}\PY{p}{]} \PY{o}{/}\PY{o}{=} \PY{n}{sp}\PY{o}{.}\PY{n}{sqrt}\PY{p}{(}\PY{l+m+mi}{4}\PY{o}{*}\PY{n}{M\PYZus{}PI}\PY{p}{)}\PY{o}{*}\PY{n}{alpha}\PY{o}{*}\PY{n}{alpha}
         
         \PY{c+c1}{\PYZsh{} Now, we\PYZsq{}ll fill out the rest of the four\PYZhy{}vector with b\PYZca{}i\PYZus{}\PYZob{},j\PYZcb{} that we derived above.}
         \PY{k}{for} \PY{n}{i} \PY{o+ow}{in} \PY{n+nb}{range}\PY{p}{(}\PY{n}{DIM}\PY{p}{)}\PY{p}{:}
             \PY{k}{for} \PY{n}{j} \PY{o+ow}{in} \PY{n+nb}{range}\PY{p}{(}\PY{n}{DIM}\PY{p}{)}\PY{p}{:}
                 \PY{n}{smallb4U\PYZus{}dD}\PY{p}{[}\PY{n}{i}\PY{o}{+}\PY{l+m+mi}{1}\PY{p}{]}\PY{p}{[}\PY{n}{j}\PY{o}{+}\PY{l+m+mi}{1}\PY{p}{]} \PY{o}{=} \PY{n}{smallbU\PYZus{}dD}\PY{p}{[}\PY{n}{i}\PY{p}{]}\PY{p}{[}\PY{n}{j}\PY{p}{]}
\end{Verbatim}

    Now we can calculate
\[\left(b^2\right)_{,j} = g_{\mu\nu,j} b^\mu b^\nu + 2 g_{\mu\nu} b^\mu_{,j} b^\nu.\]

    \begin{Verbatim}[commandchars=\\\{\}]
{\color{incolor}In [{\color{incolor}12}]:} \PY{n}{smallb2\PYZus{}dD} \PY{o}{=} \PY{n}{ixp}\PY{o}{.}\PY{n}{zerorank1}\PY{p}{(}\PY{p}{)}
         \PY{k}{for} \PY{n}{j} \PY{o+ow}{in} \PY{n+nb}{range}\PY{p}{(}\PY{n}{DIM}\PY{p}{)}\PY{p}{:}
             \PY{k}{for} \PY{n}{mu} \PY{o+ow}{in} \PY{n+nb}{range}\PY{p}{(}\PY{l+m+mi}{4}\PY{p}{)}\PY{p}{:}
                 \PY{k}{for} \PY{n}{nu} \PY{o+ow}{in} \PY{n+nb}{range}\PY{p}{(}\PY{l+m+mi}{4}\PY{p}{)}\PY{p}{:}
                     \PY{c+c1}{\PYZsh{}   g\PYZus{}\PYZob{}\PYZbs{}mu\PYZbs{}nu,j\PYZcb{} b\PYZca{}\PYZbs{}mu b\PYZca{}\PYZbs{}nu}
                     \PY{c+c1}{\PYZsh{} + 2 g\PYZus{}\PYZob{}\PYZbs{}mu\PYZbs{}nu\PYZcb{} b\PYZca{}\PYZbs{}mu\PYZus{}\PYZob{},j\PYZcb{} b\PYZca{}\PYZbs{}nu}
                     \PY{n}{smallb2\PYZus{}dD}\PY{p}{[}\PY{n}{j}\PY{p}{]} \PY{o}{=}   \PY{n}{g4DDdD}\PY{p}{[}\PY{n}{mu}\PY{p}{]}\PY{p}{[}\PY{n}{nu}\PY{p}{]}\PY{p}{[}\PY{n}{j}\PY{o}{+}\PY{l+m+mi}{1}\PY{p}{]}\PY{o}{*}\PY{n}{smallb4U}\PY{p}{[}\PY{n}{mu}\PY{p}{]}\PY{o}{*}\PY{n}{smallb4U}\PY{p}{[}\PY{n}{nu}\PY{p}{]} \PYZbs{}
                                     \PY{o}{+} \PY{l+m+mi}{2}\PY{o}{*}\PY{n}{g4DD}\PY{p}{[}\PY{n}{mu}\PY{p}{]}\PY{p}{[}\PY{n}{nu}\PY{p}{]}\PY{o}{*}\PY{n}{smallb4U\PYZus{}dD}\PY{p}{[}\PY{n}{mu}\PY{p}{]}\PY{p}{[}\PY{n}{j}\PY{o}{+}\PY{l+m+mi}{1}\PY{p}{]}\PY{o}{*}\PY{n}{smallb4U}\PY{p}{[}\PY{n}{nu}\PY{p}{]}
\end{Verbatim}

    We will also need derivatives of the spatial part of the inverse
four-metric: since
\(g^{ij} = \gamma^{ij} - \frac{\beta^i \beta^j}{\alpha^2}\)
(\href{https://arxiv.org/pdf/gr-qc/0703035.pdf}{Gourgoulhon, eq. 4.49}),

\begin{align}
g^{ij}_{\ ,k} &= \gamma^{ij}_{\ ,k} - \frac{\alpha^2 \partial_k (\beta^i \beta^j) - \beta^i \beta^j \partial_k \alpha^2}{(\alpha^2)^2} \\
&= \gamma^{ij}_{\ ,k} - \frac{\alpha^2\beta^i \beta^j_{,k}+\alpha^2\beta^i_{,k} \beta^j-2\beta^i \beta^j \alpha \alpha_{,k}}{\alpha^4}. \\
&= \gamma^{ij}_{\ ,k} - \frac{\alpha\beta^i \beta^j_{,k}+\alpha\beta^i_{,k} \beta^j-2\beta^i \beta^j \alpha_{,k}}{\alpha^3} \\
g^{ij}_{\ ,k} &= \underbrace{\gamma^{ij}_{\ ,k}}_{\rm Term\ 1} - \underbrace{\frac{\beta^i \beta^j_{,k}}{\alpha^2}}_{\rm Term\ 2} - \underbrace{\frac{\beta^i_{,k} \beta^j}{\alpha^2}}_{\rm Term\ 3} + \underbrace{\frac{2\beta^i \beta^j \alpha_{,k}}{\alpha^3}}_{\rm Term\ 4}. \\
\end{align}

    \begin{Verbatim}[commandchars=\\\{\}]
{\color{incolor}In [{\color{incolor}13}]:} \PY{c+c1}{\PYZsh{} Step 6d: Construct derivatives of the spatial components of g\PYZca{}\PYZob{}ij\PYZcb{}}
         \PY{n}{gammaUU\PYZus{}dD} \PY{o}{=} \PY{n}{ixp}\PY{o}{.}\PY{n}{declarerank3}\PY{p}{(}\PY{l+s+s2}{\PYZdq{}}\PY{l+s+s2}{gammaUU\PYZus{}dD}\PY{l+s+s2}{\PYZdq{}}\PY{p}{,}\PY{l+s+s2}{\PYZdq{}}\PY{l+s+s2}{sym01}\PY{l+s+s2}{\PYZdq{}}\PY{p}{)}
         
         \PY{c+c1}{\PYZsh{} The spatial derivatives of the spatial components of the four metric:}
         \PY{c+c1}{\PYZsh{} Term 1: \PYZbs{}gamma\PYZca{}\PYZob{}ij\PYZcb{}\PYZus{}\PYZob{}\PYZbs{} ,k\PYZcb{}}
         \PY{n}{gSpatialUU\PYZus{}dD} \PY{o}{=} \PY{n}{ixp}\PY{o}{.}\PY{n}{zerorank3}\PY{p}{(}\PY{p}{)}
         \PY{k}{for} \PY{n}{i} \PY{o+ow}{in} \PY{n+nb}{range}\PY{p}{(}\PY{n}{DIM}\PY{p}{)}\PY{p}{:}
             \PY{k}{for} \PY{n}{j} \PY{o+ow}{in} \PY{n+nb}{range}\PY{p}{(}\PY{n}{DIM}\PY{p}{)}\PY{p}{:}
                 \PY{k}{for} \PY{n}{k} \PY{o+ow}{in} \PY{n+nb}{range}\PY{p}{(}\PY{n}{DIM}\PY{p}{)}\PY{p}{:}
                     \PY{n}{gSpatialUU\PYZus{}dD}\PY{p}{[}\PY{n}{i}\PY{p}{]}\PY{p}{[}\PY{n}{j}\PY{p}{]}\PY{p}{[}\PY{n}{k}\PY{p}{]} \PY{o}{=} \PY{n}{gammaUU\PYZus{}dD}\PY{p}{[}\PY{n}{i}\PY{p}{]}\PY{p}{[}\PY{n}{j}\PY{p}{]}\PY{p}{[}\PY{n}{k}\PY{p}{]}
         
         \PY{c+c1}{\PYZsh{} Term 2: \PYZhy{} \PYZbs{}beta\PYZca{}i \PYZbs{}beta\PYZca{}j\PYZus{}\PYZob{},k\PYZcb{} / \PYZbs{}alpha\PYZca{}2}
         \PY{k}{for} \PY{n}{i} \PY{o+ow}{in} \PY{n+nb}{range}\PY{p}{(}\PY{n}{DIM}\PY{p}{)}\PY{p}{:}
             \PY{k}{for} \PY{n}{j} \PY{o+ow}{in} \PY{n+nb}{range}\PY{p}{(}\PY{n}{DIM}\PY{p}{)}\PY{p}{:}
                 \PY{k}{for} \PY{n}{k} \PY{o+ow}{in} \PY{n+nb}{range}\PY{p}{(}\PY{n}{DIM}\PY{p}{)}\PY{p}{:}
                     \PY{n}{gSpatialUU\PYZus{}dD}\PY{p}{[}\PY{n}{i}\PY{p}{]}\PY{p}{[}\PY{n}{j}\PY{p}{]}\PY{p}{[}\PY{n}{k}\PY{p}{]} \PY{o}{+}\PY{o}{=} \PY{o}{\PYZhy{}}\PY{n}{betaU}\PY{p}{[}\PY{n}{i}\PY{p}{]}\PY{o}{*}\PY{n}{betaU\PYZus{}dD}\PY{p}{[}\PY{n}{j}\PY{p}{]}\PY{p}{[}\PY{n}{k}\PY{p}{]}\PY{o}{/}\PY{n}{alpha}\PY{o}{*}\PY{o}{*}\PY{l+m+mi}{2}
         
         \PY{c+c1}{\PYZsh{} Term 3: \PYZhy{} \PYZbs{}beta\PYZca{}i\PYZus{}\PYZob{},k\PYZcb{} \PYZbs{}beta\PYZca{}j / \PYZbs{}alpha\PYZca{}2}
         \PY{k}{for} \PY{n}{i} \PY{o+ow}{in} \PY{n+nb}{range}\PY{p}{(}\PY{n}{DIM}\PY{p}{)}\PY{p}{:}
             \PY{k}{for} \PY{n}{j} \PY{o+ow}{in} \PY{n+nb}{range}\PY{p}{(}\PY{n}{DIM}\PY{p}{)}\PY{p}{:}
                 \PY{k}{for} \PY{n}{k} \PY{o+ow}{in} \PY{n+nb}{range}\PY{p}{(}\PY{n}{DIM}\PY{p}{)}\PY{p}{:}
                     \PY{n}{gSpatialUU\PYZus{}dD}\PY{p}{[}\PY{n}{i}\PY{p}{]}\PY{p}{[}\PY{n}{j}\PY{p}{]}\PY{p}{[}\PY{n}{k}\PY{p}{]} \PY{o}{+}\PY{o}{=} \PY{o}{\PYZhy{}}\PY{n}{betaU\PYZus{}dD}\PY{p}{[}\PY{n}{i}\PY{p}{]}\PY{p}{[}\PY{n}{k}\PY{p}{]}\PY{o}{*}\PY{n}{betaU}\PY{p}{[}\PY{n}{j}\PY{p}{]}\PY{o}{/}\PY{n}{alpha}\PY{o}{*}\PY{o}{*}\PY{l+m+mi}{2}
         
         \PY{c+c1}{\PYZsh{} Term 4: 2\PYZbs{}beta\PYZca{}i \PYZbs{}beta\PYZca{}j \PYZbs{}alpha\PYZus{}\PYZob{},k\PYZcb{}\PYZbs{}alpha\PYZca{}3}
         \PY{k}{for} \PY{n}{i} \PY{o+ow}{in} \PY{n+nb}{range}\PY{p}{(}\PY{n}{DIM}\PY{p}{)}\PY{p}{:}
             \PY{k}{for} \PY{n}{j} \PY{o+ow}{in} \PY{n+nb}{range}\PY{p}{(}\PY{n}{DIM}\PY{p}{)}\PY{p}{:}
                 \PY{k}{for} \PY{n}{k} \PY{o+ow}{in} \PY{n+nb}{range}\PY{p}{(}\PY{n}{DIM}\PY{p}{)}\PY{p}{:}
                     \PY{n}{gSpatialUU\PYZus{}dD}\PY{p}{[}\PY{n}{i}\PY{p}{]}\PY{p}{[}\PY{n}{j}\PY{p}{]}\PY{p}{[}\PY{n}{k}\PY{p}{]} \PY{o}{+}\PY{o}{=} \PY{l+m+mi}{2}\PY{o}{*}\PY{n}{betaU}\PY{p}{[}\PY{n}{i}\PY{p}{]}\PY{o}{*}\PY{n}{betaU}\PY{p}{[}\PY{n}{j}\PY{p}{]}\PY{o}{*}\PY{n}{alpha\PYZus{}dD}\PY{p}{[}\PY{n}{k}\PY{p}{]}\PY{o}{/}\PY{n}{alpha}\PY{o}{*}\PY{o}{*}\PY{l+m+mi}{3}
\end{Verbatim}

     {[}Back to the Section \ref{table2}{]}

So, we can now put it all together:

\begin{align}
\partial_j  T^{j}_{{\rm EM} i} &= \gamma_{ki,j} T^{kj}_{\rm EM} + \gamma_{ki} \left(\left(b^2\right)_{,j} u^j u^k + b^2 u^j_{,j} u^k + b^2 u^j u^k_{,j} + \frac{1}{2} \left(b^2\right)_{,j} g^{jk} + \frac{b^2}{2} g^{jk}_{\ ,j} + b^j_{,j} b^k + b^j b^k_{,j}\right).
\end{align}

It should be noted that due to the way our indexing conventions have
have fallen, the Python indices for \(T^{ij}_{\rm EM}\), \(g^{ij}\),
\(b^i\) and \(b_i\) will need to be incremented to correctly use the
spatial components. We will also quickly rearrange the terms of the
expression to better mimic the loop structure we will need to create:

\begin{align}
\partial_j  T^{j}_{{\rm EM} i} =& \ 
\underbrace{\gamma_{ki,j} T^{kj}_{\rm EM}}_{\rm Term\ 1} \\
& + \underbrace{\gamma_{ki} \left( b^2 u^j_{,j} u^k + b^2 u^j u^k_{,j} + \frac{b^2}{2} g^{jk}_{\ ,j} + b^j_{,j} b^k + b^j b^k_{,j} \right)}_{\rm Term\ 2} \\
& + \underbrace{\gamma_{ki} \left( \left(b^2\right)_{,j} u^j u^k + \frac{1}{2} \left(b^2\right)_{,j} g^{jk} \right).}_{\rm Term\ 3} \\
\end{align}

We will now construct this term by term. Term 1 is straightforward:
\[{\rm Term\ 1} = \gamma_{ki,j} T^{kj}_{\rm EM}.\]

    \begin{Verbatim}[commandchars=\\\{\}]
{\color{incolor}In [{\color{incolor}14}]:} \PY{c+c1}{\PYZsh{} Step 6e: Construct TEMUD\PYZus{}dD\PYZus{}contracted itself}
         \PY{c+c1}{\PYZsh{} Step 6e.i}
         \PY{n}{TEMUD\PYZus{}dD\PYZus{}contracted} \PY{o}{=} \PY{n}{ixp}\PY{o}{.}\PY{n}{zerorank1}\PY{p}{(}\PY{p}{)}
         \PY{k}{for} \PY{n}{i} \PY{o+ow}{in} \PY{n+nb}{range}\PY{p}{(}\PY{n}{DIM}\PY{p}{)}\PY{p}{:}
             \PY{k}{for} \PY{n}{j} \PY{o+ow}{in} \PY{n+nb}{range}\PY{p}{(}\PY{n}{DIM}\PY{p}{)}\PY{p}{:}
                 \PY{k}{for} \PY{n}{k} \PY{o+ow}{in} \PY{n+nb}{range}\PY{p}{(}\PY{n}{DIM}\PY{p}{)}\PY{p}{:}
                     \PY{n}{TEMUD\PYZus{}dD\PYZus{}contracted}\PY{p}{[}\PY{n}{i}\PY{p}{]} \PY{o}{+}\PY{o}{=} \PY{n}{gammaDD\PYZus{}dD}\PY{p}{[}\PY{n}{k}\PY{p}{]}\PY{p}{[}\PY{n}{i}\PY{p}{]}\PY{p}{[}\PY{n}{j}\PY{p}{]} \PY{o}{*} \PY{n}{T4EMUU}\PY{p}{[}\PY{n}{k}\PY{o}{+}\PY{l+m+mi}{1}\PY{p}{]}\PY{p}{[}\PY{n}{j}\PY{o}{+}\PY{l+m+mi}{1}\PY{p}{]}
\end{Verbatim}

    We will now add
\[{\rm Term\ 2} = \gamma_{ki} \left( \underbrace{b^2 u^j_{,j} u^k}_{\rm Term\ 2a} + \underbrace{b^2 u^j u^k_{,j}}_{\rm Term\ 2b} + \underbrace{\frac{b^2}{2} g^{jk}_{\ ,j}}_{\rm Term\ 2c} + \underbrace{b^j_{,j} b^k}_{\rm Term\ 2d} + \underbrace{b^j b^k_{,j}}_{\rm Term\ 2e} \right)\]
to \(\partial_j T^{j}_{{\rm EM} i}\). These are the terms that involve
contractions over \(k\) (but no metric derivatives like Term 1 had).

    \begin{Verbatim}[commandchars=\\\{\}]
{\color{incolor}In [{\color{incolor}15}]:} \PY{c+c1}{\PYZsh{} Step 6e.ii}
         \PY{k}{for} \PY{n}{i} \PY{o+ow}{in} \PY{n+nb}{range}\PY{p}{(}\PY{n}{DIM}\PY{p}{)}\PY{p}{:}
             \PY{k}{for} \PY{n}{j} \PY{o+ow}{in} \PY{n+nb}{range}\PY{p}{(}\PY{n}{DIM}\PY{p}{)}\PY{p}{:}
                 \PY{k}{for} \PY{n}{k} \PY{o+ow}{in} \PY{n+nb}{range}\PY{p}{(}\PY{n}{DIM}\PY{p}{)}\PY{p}{:}
                     \PY{c+c1}{\PYZsh{} Term 2a: \PYZbs{}gamma\PYZus{}\PYZob{}ki\PYZcb{} b\PYZca{}2 u\PYZca{}j\PYZus{}\PYZob{},j\PYZcb{} u\PYZca{}k}
                     \PY{n}{TEMUD\PYZus{}dD\PYZus{}contracted}\PY{p}{[}\PY{n}{i}\PY{p}{]} \PY{o}{+}\PY{o}{=} \PY{n}{gammaDD}\PY{p}{[}\PY{n}{k}\PY{p}{]}\PY{p}{[}\PY{n}{i}\PY{p}{]}\PY{o}{*}\PY{n}{smallb2}\PY{o}{*}\PY{n}{uU\PYZus{}dD}\PY{p}{[}\PY{n}{j}\PY{p}{]}\PY{p}{[}\PY{n}{j}\PY{p}{]}\PY{o}{*}\PY{n}{uU}\PY{p}{[}\PY{n}{k}\PY{p}{]}
         
         \PY{k}{for} \PY{n}{i} \PY{o+ow}{in} \PY{n+nb}{range}\PY{p}{(}\PY{n}{DIM}\PY{p}{)}\PY{p}{:}
             \PY{k}{for} \PY{n}{j} \PY{o+ow}{in} \PY{n+nb}{range}\PY{p}{(}\PY{n}{DIM}\PY{p}{)}\PY{p}{:}
                 \PY{k}{for} \PY{n}{k} \PY{o+ow}{in} \PY{n+nb}{range}\PY{p}{(}\PY{n}{DIM}\PY{p}{)}\PY{p}{:}
                     \PY{c+c1}{\PYZsh{} Term 2b: \PYZbs{}gamma\PYZus{}\PYZob{}ki\PYZcb{} b\PYZca{}2 u\PYZca{}j u\PYZca{}k\PYZus{}\PYZob{},j\PYZcb{}}
                     \PY{n}{TEMUD\PYZus{}dD\PYZus{}contracted}\PY{p}{[}\PY{n}{i}\PY{p}{]} \PY{o}{+}\PY{o}{=} \PY{n}{gammaDD}\PY{p}{[}\PY{n}{k}\PY{p}{]}\PY{p}{[}\PY{n}{i}\PY{p}{]}\PY{o}{*}\PY{n}{smallb2}\PY{o}{*}\PY{n}{uU}\PY{p}{[}\PY{n}{j}\PY{p}{]}\PY{o}{*}\PY{n}{uU\PYZus{}dD}\PY{p}{[}\PY{n}{k}\PY{p}{]}\PY{p}{[}\PY{n}{j}\PY{p}{]}
         
         \PY{k}{for} \PY{n}{i} \PY{o+ow}{in} \PY{n+nb}{range}\PY{p}{(}\PY{n}{DIM}\PY{p}{)}\PY{p}{:}
             \PY{k}{for} \PY{n}{j} \PY{o+ow}{in} \PY{n+nb}{range}\PY{p}{(}\PY{n}{DIM}\PY{p}{)}\PY{p}{:}
                 \PY{k}{for} \PY{n}{k} \PY{o+ow}{in} \PY{n+nb}{range}\PY{p}{(}\PY{n}{DIM}\PY{p}{)}\PY{p}{:}
                     \PY{c+c1}{\PYZsh{} Term 2c: \PYZbs{}gamma\PYZus{}\PYZob{}ki\PYZcb{} b\PYZca{}2 / 2 g\PYZca{}\PYZob{}jk\PYZcb{}\PYZus{}\PYZob{},j\PYZcb{}}
                     \PY{n}{TEMUD\PYZus{}dD\PYZus{}contracted}\PY{p}{[}\PY{n}{i}\PY{p}{]} \PY{o}{+}\PY{o}{=} \PY{n}{gammaDD}\PY{p}{[}\PY{n}{k}\PY{p}{]}\PY{p}{[}\PY{n}{i}\PY{p}{]}\PY{o}{*}\PY{n}{smallb2}\PY{o}{*}\PY{n}{gSpatialUU\PYZus{}dD}\PY{p}{[}\PY{n}{j}\PY{p}{]}\PY{p}{[}\PY{n}{k}\PY{p}{]}\PY{p}{[}\PY{n}{j}\PY{p}{]}\PY{o}{/}\PY{l+m+mi}{2}
         
         \PY{k}{for} \PY{n}{i} \PY{o+ow}{in} \PY{n+nb}{range}\PY{p}{(}\PY{n}{DIM}\PY{p}{)}\PY{p}{:}
             \PY{k}{for} \PY{n}{j} \PY{o+ow}{in} \PY{n+nb}{range}\PY{p}{(}\PY{n}{DIM}\PY{p}{)}\PY{p}{:}
                 \PY{k}{for} \PY{n}{k} \PY{o+ow}{in} \PY{n+nb}{range}\PY{p}{(}\PY{n}{DIM}\PY{p}{)}\PY{p}{:}
                     \PY{c+c1}{\PYZsh{} Term 2d: \PYZbs{}gamma\PYZus{}\PYZob{}ki\PYZcb{} b\PYZca{}j\PYZus{}\PYZob{},j\PYZcb{} b\PYZca{}k}
                     \PY{n}{TEMUD\PYZus{}dD\PYZus{}contracted}\PY{p}{[}\PY{n}{i}\PY{p}{]} \PY{o}{+}\PY{o}{=} \PY{n}{gammaDD}\PY{p}{[}\PY{n}{k}\PY{p}{]}\PY{p}{[}\PY{n}{i}\PY{p}{]}\PY{o}{*}\PY{n}{smallbU\PYZus{}dD}\PY{p}{[}\PY{n}{j}\PY{p}{]}\PY{p}{[}\PY{n}{j}\PY{p}{]}\PY{o}{*}\PY{n}{smallb4U}\PY{p}{[}\PY{n}{k}\PY{o}{+}\PY{l+m+mi}{1}\PY{p}{]}
         
         \PY{k}{for} \PY{n}{i} \PY{o+ow}{in} \PY{n+nb}{range}\PY{p}{(}\PY{n}{DIM}\PY{p}{)}\PY{p}{:}
             \PY{k}{for} \PY{n}{j} \PY{o+ow}{in} \PY{n+nb}{range}\PY{p}{(}\PY{n}{DIM}\PY{p}{)}\PY{p}{:}
                 \PY{k}{for} \PY{n}{k} \PY{o+ow}{in} \PY{n+nb}{range}\PY{p}{(}\PY{n}{DIM}\PY{p}{)}\PY{p}{:}
                     \PY{c+c1}{\PYZsh{} Term 2e: \PYZbs{}gamma\PYZus{}\PYZob{}ki\PYZcb{} b\PYZca{}j b\PYZca{}k\PYZus{}\PYZob{},j\PYZcb{}}
                     \PY{n}{TEMUD\PYZus{}dD\PYZus{}contracted}\PY{p}{[}\PY{n}{i}\PY{p}{]} \PY{o}{+}\PY{o}{=} \PY{n}{gammaDD}\PY{p}{[}\PY{n}{k}\PY{p}{]}\PY{p}{[}\PY{n}{i}\PY{p}{]}\PY{o}{*}\PY{n}{smallb4U}\PY{p}{[}\PY{n}{j}\PY{o}{+}\PY{l+m+mi}{1}\PY{p}{]}\PY{o}{*}\PY{n}{smallbU\PYZus{}dD}\PY{p}{[}\PY{n}{k}\PY{p}{]}\PY{p}{[}\PY{n}{j}\PY{p}{]}
\end{Verbatim}

    Now, we will add
\[{\rm Term\ 3} = \gamma_{ki} \left( \underbrace{\left(b^2\right)_{,j} u^j u^k}_{\rm Term\ 3a} + \underbrace{\frac{1}{2} \left(b^2\right)_{,j} g^{jk}}_{\rm Term\ 3b} \right).\]

    \begin{Verbatim}[commandchars=\\\{\}]
{\color{incolor}In [{\color{incolor}16}]:} \PY{c+c1}{\PYZsh{} Step 6e.iii}
         \PY{k}{for} \PY{n}{i} \PY{o+ow}{in} \PY{n+nb}{range}\PY{p}{(}\PY{n}{DIM}\PY{p}{)}\PY{p}{:}
             \PY{k}{for} \PY{n}{j} \PY{o+ow}{in} \PY{n+nb}{range}\PY{p}{(}\PY{n}{DIM}\PY{p}{)}\PY{p}{:}
                 \PY{k}{for} \PY{n}{k} \PY{o+ow}{in} \PY{n+nb}{range}\PY{p}{(}\PY{n}{DIM}\PY{p}{)}\PY{p}{:}
                     \PY{k}{for} \PY{n}{l} \PY{o+ow}{in} \PY{n+nb}{range}\PY{p}{(}\PY{n}{DIM}\PY{p}{)}\PY{p}{:}
                         \PY{c+c1}{\PYZsh{} Term 3a: \PYZbs{}gamma\PYZus{}\PYZob{}ki\PYZcb{} ( b\PYZca{}2 )\PYZus{}\PYZob{},j\PYZcb{} u\PYZca{}j u\PYZca{}k}
                         \PY{n}{TEMUD\PYZus{}dD\PYZus{}contracted}\PY{p}{[}\PY{n}{i}\PY{p}{]} \PY{o}{+}\PY{o}{=} \PY{n}{gammaDD}\PY{p}{[}\PY{n}{k}\PY{p}{]}\PY{p}{[}\PY{n}{i}\PY{p}{]}\PY{o}{*}\PY{n}{smallb2\PYZus{}dD}\PY{p}{[}\PY{n}{j}\PY{p}{]}\PY{o}{*}\PY{n}{uU}\PY{p}{[}\PY{n}{j}\PY{p}{]}\PY{o}{*}\PY{n}{uU}\PY{p}{[}\PY{n}{k}\PY{p}{]}
         
         \PY{k}{for} \PY{n}{i} \PY{o+ow}{in} \PY{n+nb}{range}\PY{p}{(}\PY{n}{DIM}\PY{p}{)}\PY{p}{:}
             \PY{k}{for} \PY{n}{j} \PY{o+ow}{in} \PY{n+nb}{range}\PY{p}{(}\PY{n}{DIM}\PY{p}{)}\PY{p}{:}
                 \PY{k}{for} \PY{n}{k} \PY{o+ow}{in} \PY{n+nb}{range}\PY{p}{(}\PY{n}{DIM}\PY{p}{)}\PY{p}{:}
                     \PY{k}{for} \PY{n}{l} \PY{o+ow}{in} \PY{n+nb}{range}\PY{p}{(}\PY{n}{DIM}\PY{p}{)}\PY{p}{:}
                         \PY{c+c1}{\PYZsh{} Term 3b: \PYZbs{}gamma\PYZus{}\PYZob{}ki\PYZcb{} ( b\PYZca{}2 )\PYZus{}\PYZob{},j\PYZcb{} g\PYZca{}\PYZob{}jk\PYZcb{} / 2}
                         \PY{n}{TEMUD\PYZus{}dD\PYZus{}contracted}\PY{p}{[}\PY{n}{i}\PY{p}{]} \PY{o}{+}\PY{o}{=} \PY{n}{gammaDD}\PY{p}{[}\PY{n}{k}\PY{p}{]}\PY{p}{[}\PY{n}{i}\PY{p}{]}\PY{o}{*}\PY{n}{smallb2\PYZus{}dD}\PY{p}{[}\PY{n}{j}\PY{p}{]}\PY{o}{*}\PY{n}{g4DD}\PY{p}{[}\PY{n}{j}\PY{p}{]}\PY{p}{[}\PY{n}{k}\PY{p}{]}
\end{Verbatim}

    \section{\texorpdfstring{Evolution equation for
\(\tilde{S}_i\)}{Evolution equation for \textbackslash{}tilde\{S\}\_i}}\label{evolution-equation-for-tildes_i}

\subsection{\texorpdfstring{Step 7: Construct the evolution equation for
\(\tilde{S}_i\)}{Step 7: Construct the evolution equation for \textbackslash{}tilde\{S\}\_i}}\label{step-7-construct-the-evolution-equation-for-tildes_i}

\[\label{step7}\]

{[}Back to Section \ref{top}{]}

Finally, we will return our attention to the time evolution equation
(from eq. 13 of the \href{https://arxiv.org/pdf/1704.00599.pdf}{original
paper}),

\begin{align}
\partial_t \tilde{S}_i &= - \partial_j \left( \alpha \sqrt{\gamma} T^j_{{\rm EM} i} \right) + \frac{1}{2} \alpha \sqrt{\gamma} T^{\mu \nu}_{\rm EM} \partial_i g_{\mu \nu} \\
                       &= -T^j_{{\rm EM} i} \partial_j (\alpha \sqrt{\gamma}) - \alpha \sqrt{\gamma} \partial_j T^j_{{\rm EM} i} + \frac{1}{2} \alpha \sqrt{\gamma} T^{\mu \nu}_{\rm EM} \partial_i g_{\mu \nu} \\
                       &= \underbrace{\frac{1}{2} \alpha \sqrt{\gamma} T^{\mu \nu}_{\rm EM} \partial_i g_{\mu \nu}}_{\rm Term\ 1} - \underbrace{\alpha \sqrt{\gamma} \partial_j T^j_{{\rm EM} i}}_{\rm Term\ 2} - \underbrace{\gamma_{ik} T^{kj}_{\rm EM} \partial_j (\alpha \sqrt{\gamma})}_{\rm Term\ 3} .
\end{align}

We construct the first term separately at first, to reduce the
complication of dealing with mixed Greek and Latin indices. Then we will
take derivatives of \(\alpha \sqrt{\gamma}\).

    \begin{Verbatim}[commandchars=\\\{\}]
{\color{incolor}In [{\color{incolor} }]:} \PY{c+c1}{\PYZsh{} Step 7: Construct the evolution equation for \PYZbs{}tilde\PYZob{}S\PYZcb{}\PYZus{}i}
        \PY{c+c1}{\PYZsh{} Here, we set up the necessary machinery to take FD derivatives of alpha * sqrt(gamma)}
        \PY{n}{alpsqrtgam} \PY{o}{=} \PY{n}{gri}\PY{o}{.}\PY{n}{register\PYZus{}gridfunctions}\PY{p}{(}\PY{l+s+s2}{\PYZdq{}}\PY{l+s+s2}{AUX}\PY{l+s+s2}{\PYZdq{}}\PY{p}{,}\PY{l+s+s2}{\PYZdq{}}\PY{l+s+s2}{alpsqrtgam}\PY{l+s+s2}{\PYZdq{}}\PY{p}{)}
        \PY{n}{alpsqrtgam} \PY{o}{=} \PY{n}{alpha}\PY{o}{*}\PY{n}{sp}\PY{o}{.}\PY{n}{sqrt}\PY{p}{(}\PY{n}{gammadet}\PY{p}{)}
        \PY{n}{alpsqrtgam\PYZus{}dD} \PY{o}{=} \PY{n}{ixp}\PY{o}{.}\PY{n}{declarerank1}\PY{p}{(}\PY{l+s+s2}{\PYZdq{}}\PY{l+s+s2}{alpsqrtgam\PYZus{}dD}\PY{l+s+s2}{\PYZdq{}}\PY{p}{)}
        
        \PY{n}{Stilde\PYZus{}rhsD} \PY{o}{=} \PY{n}{ixp}\PY{o}{.}\PY{n}{zerorank1}\PY{p}{(}\PY{p}{)}
        \PY{c+c1}{\PYZsh{} The first term: \PYZbs{}alpha \PYZbs{}sqrt\PYZob{}\PYZbs{}gamma\PYZcb{} T\PYZca{}\PYZob{}\PYZbs{}mu \PYZbs{}nu\PYZcb{}\PYZus{}\PYZob{}\PYZbs{}rm EM\PYZcb{} \PYZbs{}partial\PYZus{}i g\PYZus{}\PYZob{}\PYZbs{}mu \PYZbs{}nu\PYZcb{} / 2}
        \PY{k}{for} \PY{n}{i} \PY{o+ow}{in} \PY{n+nb}{range}\PY{p}{(}\PY{n}{DIM}\PY{p}{)}\PY{p}{:}
            \PY{k}{for} \PY{n}{mu} \PY{o+ow}{in} \PY{n+nb}{range}\PY{p}{(}\PY{l+m+mi}{4}\PY{p}{)}\PY{p}{:}
                \PY{k}{for} \PY{n}{nu} \PY{o+ow}{in} \PY{n+nb}{range}\PY{p}{(}\PY{l+m+mi}{4}\PY{p}{)}\PY{p}{:}
                    \PY{n}{Stilde\PYZus{}rhsD}\PY{p}{[}\PY{n}{i}\PY{p}{]} \PY{o}{+}\PY{o}{=} \PY{n}{alpsqrtgam} \PY{o}{*} \PY{n}{T4EMUU}\PY{p}{[}\PY{n}{mu}\PY{p}{]}\PY{p}{[}\PY{n}{nu}\PY{p}{]} \PY{o}{*} \PY{n}{g4DDdD}\PY{p}{[}\PY{n}{mu}\PY{p}{]}\PY{p}{[}\PY{n}{nu}\PY{p}{]}\PY{p}{[}\PY{n}{i}\PY{o}{+}\PY{l+m+mi}{1}\PY{p}{]} \PY{o}{/} \PY{l+m+mi}{2}
        
        \PY{c+c1}{\PYZsh{} The second term: \PYZbs{}alpha \PYZbs{}sqrt\PYZob{}\PYZbs{}gamma\PYZcb{} \PYZbs{}partial\PYZus{}j T\PYZca{}j\PYZus{}\PYZob{}\PYZob{}\PYZbs{}rm EM\PYZcb{} i\PYZcb{}}
        \PY{k}{for} \PY{n}{i} \PY{o+ow}{in} \PY{n+nb}{range}\PY{p}{(}\PY{n}{DIM}\PY{p}{)}\PY{p}{:}
            \PY{n}{Stilde\PYZus{}rhsD}\PY{p}{[}\PY{n}{i}\PY{p}{]} \PY{o}{+}\PY{o}{=} \PY{o}{\PYZhy{}}\PY{n}{alpsqrtgam} \PY{o}{*} \PY{n}{TEMUD\PYZus{}dD\PYZus{}contracted}\PY{p}{[}\PY{n}{i}\PY{p}{]}
        
        \PY{c+c1}{\PYZsh{} The third term: \PYZbs{}gamma\PYZus{}\PYZob{}ik\PYZcb{} T\PYZca{}\PYZob{}kj\PYZcb{}\PYZus{}\PYZob{}\PYZbs{}rm EM\PYZcb{} \PYZbs{}partial\PYZus{}j (\PYZbs{}alpha \PYZbs{}sqrt\PYZob{}\PYZbs{}gamma\PYZcb{})}
        \PY{k}{for} \PY{n}{i} \PY{o+ow}{in} \PY{n+nb}{range}\PY{p}{(}\PY{n}{DIM}\PY{p}{)}\PY{p}{:}
            \PY{k}{for} \PY{n}{j} \PY{o+ow}{in} \PY{n+nb}{range}\PY{p}{(}\PY{n}{DIM}\PY{p}{)}\PY{p}{:}
                \PY{k}{for} \PY{n}{k} \PY{o+ow}{in} \PY{n+nb}{range}\PY{p}{(}\PY{n}{DIM}\PY{p}{)}\PY{p}{:}
                    \PY{n}{Stilde\PYZus{}rhsD}\PY{p}{[}\PY{n}{i}\PY{p}{]} \PY{o}{+}\PY{o}{=} \PY{o}{\PYZhy{}}\PY{n}{gammaDD}\PY{p}{[}\PY{n}{i}\PY{p}{]}\PY{p}{[}\PY{n}{k}\PY{p}{]}\PY{o}{*}\PY{n}{T4EMUU}\PY{p}{[}\PY{n}{k}\PY{p}{]}\PY{p}{[}\PY{n}{j}\PY{p}{]}\PY{o}{*}\PY{n}{alpsqrtgam\PYZus{}dD}\PY{p}{[}\PY{n}{j}\PY{p}{]}
\end{Verbatim}

    \section{\texorpdfstring{Evolution equations for \(A_i\) and
\(\Phi\)}{Evolution equations for A\_i and \textbackslash{}Phi}}\label{evolution-equations-for-a_i-and-phi}

 \#\# Step 8: Construct the evolution equations for \(A_i\) and
\(\sqrt{\gamma}\Phi\) \[\label{step8}\]

{[}Back to Section \ref{top}{]}

We will also need to evolve the vector potential \(A_i\). This evolution
is given as eq. 17 in the
\href{https://arxiv.org/pdf/1704.00599.pdf}{\(\texttt{GiRaFFE}\)} paper:
\[\boxed{\partial_t A_i = \epsilon_{ijk} v^j B^k - \partial_i (\alpha \Phi - \beta^j A_j),}\]
where \(\epsilon_{ijk} = [ijk] \sqrt{\gamma}\) is the antisymmetric
Levi-Civita tensor, the drift velocity \(v^i = u^i/u^0\), and \(\gamma\)
is the determinant of the three metric. The scalar electric potential
\(\Phi\) is also evolved by eq. 19:
\[\boxed{\partial_t [\sqrt{\gamma} \Phi] = -\partial_j (\alpha\sqrt{\gamma}A^j - \beta^j [\sqrt{\gamma} \Phi]) - \xi \alpha [\sqrt{\gamma} \Phi],}\]
with \(\xi\) chosen as a damping factor.

\subsubsection{Step 8a: Construct some useful auxiliary gridfunctions
for the other evolution
equations}\label{step-8a-construct-some-useful-auxiliary-gridfunctions-for-the-other-evolution-equations}

After declaring a some needed quantities, we will also define the
parenthetical terms that we need to take derivatives of. For \(A_i\), we
will define \[{\rm AevolParen} = \alpha \Phi - \beta^j A_j\] and for
\(\sqrt{\gamma} \Phi\), we will define

\begin{align}
{\rm PevolParenU[j]} &= \alpha\sqrt{\gamma}A^j - \beta^j [\sqrt{\gamma} \Phi] \\
&= \alpha\sqrt{\gamma\ } \gamma^{ij} A_i - \beta^j [\sqrt{\gamma} \Phi]. \\
\end{align}

    \begin{Verbatim}[commandchars=\\\{\}]
{\color{incolor}In [{\color{incolor} }]:} \PY{c+c1}{\PYZsh{} Step 8: Construct the evolution equations for A\PYZus{}i and sqrt(gamma)Phi}
        \PY{c+c1}{\PYZsh{} Step 8a: Construct some useful auxiliary gridfunctions for the other evolution equations}
        \PY{n}{xi} \PY{o}{=} \PY{n}{par}\PY{o}{.}\PY{n}{Cparameters}\PY{p}{(}\PY{l+s+s2}{\PYZdq{}}\PY{l+s+s2}{REAL}\PY{l+s+s2}{\PYZdq{}}\PY{p}{,}\PY{n}{thismodule}\PY{p}{,}\PY{l+s+s2}{\PYZdq{}}\PY{l+s+s2}{xi}\PY{l+s+s2}{\PYZdq{}}\PY{p}{)} \PY{c+c1}{\PYZsh{} The damping factor}
        
        \PY{c+c1}{\PYZsh{} Call sqrt(gamma)Phi psi6Phi}
        \PY{n}{psi6Phi} \PY{o}{=} \PY{n}{gri}\PY{o}{.}\PY{n}{register\PYZus{}gridfunctions}\PY{p}{(}\PY{l+s+s2}{\PYZdq{}}\PY{l+s+s2}{AUX}\PY{l+s+s2}{\PYZdq{}}\PY{p}{,}\PY{l+s+s2}{\PYZdq{}}\PY{l+s+s2}{psi6Phi}\PY{l+s+s2}{\PYZdq{}}\PY{p}{)}
        \PY{n}{Phi} \PY{o}{=} \PY{n}{psi6Phi} \PY{o}{/} \PY{n}{sp}\PY{o}{.}\PY{n}{sqrt}\PY{p}{(}\PY{n}{gammadet}\PY{p}{)}
        
        \PY{c+c1}{\PYZsh{} We\PYZsq{}ll define a few extra gridfunctions to avoid complicated derivatives}
        \PY{n}{AevolParen}  \PY{o}{=} \PY{n}{gri}\PY{o}{.}\PY{n}{register\PYZus{}gridfunctions}\PY{p}{(}\PY{l+s+s2}{\PYZdq{}}\PY{l+s+s2}{AUX}\PY{l+s+s2}{\PYZdq{}}\PY{p}{,}\PY{l+s+s2}{\PYZdq{}}\PY{l+s+s2}{AevolParen}\PY{l+s+s2}{\PYZdq{}}\PY{p}{)}
        \PY{n}{PevolParenU} \PY{o}{=} \PY{n}{ixp}\PY{o}{.}\PY{n}{register\PYZus{}gridfunctions\PYZus{}for\PYZus{}single\PYZus{}rank1}\PY{p}{(}\PY{l+s+s2}{\PYZdq{}}\PY{l+s+s2}{AUX}\PY{l+s+s2}{\PYZdq{}}\PY{p}{,}\PY{l+s+s2}{\PYZdq{}}\PY{l+s+s2}{PevolParenU}\PY{l+s+s2}{\PYZdq{}}\PY{p}{)}
        
        \PY{c+c1}{\PYZsh{} \PYZob{}\PYZbs{}rm AevolParen\PYZcb{} = \PYZbs{}alpha \PYZbs{}Phi \PYZhy{} \PYZbs{}beta\PYZca{}j A\PYZus{}j}
        \PY{n}{AevolParen} \PY{o}{=} \PY{n}{alpha}\PY{o}{*}\PY{n}{Phi}
        \PY{k}{for} \PY{n}{j} \PY{o+ow}{in} \PY{n+nb}{range}\PY{p}{(}\PY{n}{DIM}\PY{p}{)}\PY{p}{:}
            \PY{n}{AevolParen}     \PY{o}{=} \PY{o}{\PYZhy{}}\PY{n}{betaU}\PY{p}{[}\PY{n}{j}\PY{p}{]} \PY{o}{*} \PY{n}{AD}\PY{p}{[}\PY{n}{j}\PY{p}{]}
        
        \PY{c+c1}{\PYZsh{} \PYZob{}\PYZbs{}rm PevolParenU[j]\PYZcb{} = \PYZbs{}alpha\PYZbs{}sqrt\PYZob{}\PYZbs{}gamma\PYZcb{} \PYZbs{}gamma\PYZca{}\PYZob{}ij\PYZcb{} A\PYZus{}i \PYZhy{} \PYZbs{}beta\PYZca{}j [\PYZbs{}sqrt\PYZob{}\PYZbs{}gamma\PYZcb{} \PYZbs{}Phi]}
        \PY{k}{for} \PY{n}{j} \PY{o+ow}{in} \PY{n+nb}{range}\PY{p}{(}\PY{n}{DIM}\PY{p}{)}\PY{p}{:}
            \PY{n}{PevolParenU}\PY{p}{[}\PY{n}{j}\PY{p}{]} \PY{o}{=} \PY{o}{\PYZhy{}}\PY{n}{betaU}\PY{p}{[}\PY{n}{j}\PY{p}{]} \PY{o}{*} \PY{n}{psi6Phi}
            \PY{k}{for} \PY{n}{i} \PY{o+ow}{in} \PY{n+nb}{range}\PY{p}{(}\PY{n}{DIM}\PY{p}{)}\PY{p}{:}
                \PY{n}{PevolParenU}\PY{p}{[}\PY{n}{j}\PY{p}{]} \PY{o}{+}\PY{o}{=} \PY{n}{alpha} \PY{o}{*} \PY{n}{sp}\PY{o}{.}\PY{n}{sqrt}\PY{p}{(}\PY{n}{gammadet}\PY{p}{)} \PY{o}{*} \PY{n}{gammaUU}\PY{p}{[}\PY{n}{i}\PY{p}{]}\PY{p}{[}\PY{n}{j}\PY{p}{]} \PY{o}{*} \PY{n}{AD}\PY{p}{[}\PY{n}{i}\PY{p}{]}
        
        \PY{n}{AevolParen\PYZus{}dD}  \PY{o}{=} \PY{n}{ixp}\PY{o}{.}\PY{n}{declarerank1}\PY{p}{(}\PY{l+s+s2}{\PYZdq{}}\PY{l+s+s2}{AevolParen\PYZus{}dD}\PY{l+s+s2}{\PYZdq{}}\PY{p}{)}
        \PY{n}{PevolParenU\PYZus{}dD} \PY{o}{=} \PY{n}{ixp}\PY{o}{.}\PY{n}{declarerank2}\PY{p}{(}\PY{l+s+s2}{\PYZdq{}}\PY{l+s+s2}{PevolParenU\PYZus{}dD}\PY{l+s+s2}{\PYZdq{}}\PY{p}{,}\PY{l+s+s2}{\PYZdq{}}\PY{l+s+s2}{nosym}\PY{l+s+s2}{\PYZdq{}}\PY{p}{)}
\end{Verbatim}

    \subsubsection{\texorpdfstring{Step 8b: Construct the actual evolution
equations for \(A_i\) and
\(\sqrt{\gamma}\Phi\)}{Step 8b: Construct the actual evolution equations for A\_i and \textbackslash{}sqrt\{\textbackslash{}gamma\}\textbackslash{}Phi}}\label{step-8b-construct-the-actual-evolution-equations-for-a_i-and-sqrtgammaphi}

Now to set the evolution equations
(\href{https://arxiv.org/pdf/1704.00599.pdf}{eqs. 17 and 19}), recalling
that the drift velocity \(v^i = u^i/u^0\):

\begin{align}
\partial_t A_i &= \epsilon_{ijk} v^j B^k - \partial_i (\alpha \Phi - \beta^j A_j) \\
               &= \epsilon_{ijk} \frac{u^j}{u^0} B^k - {\rm AevolParen\_dD[i]} \\
\partial_t [\sqrt{\gamma} \Phi] &= -\partial_j (\alpha\sqrt{\gamma}A^j - \beta^j [\sqrt{\gamma} \Phi]) - \xi \alpha [\sqrt{\gamma} \Phi] \\
                                &= -{\rm PevolParenU\_dD[j][j]} - \xi \alpha [\sqrt{\gamma} \Phi]. \\
\end{align}

    \begin{Verbatim}[commandchars=\\\{\}]
{\color{incolor}In [{\color{incolor} }]:} \PY{c+c1}{\PYZsh{} Step 8b: Construct the actual evolution equations for A\PYZus{}i and sqrt(gamma)Phi}
        \PY{n}{A\PYZus{}rhsD} \PY{o}{=} \PY{n}{ixp}\PY{o}{.}\PY{n}{zerorank1}\PY{p}{(}\PY{p}{)}
        \PY{n}{psi6Phi\PYZus{}rhs} \PY{o}{=} \PY{n}{sp}\PY{o}{.}\PY{n}{sympify}\PY{p}{(}\PY{l+m+mi}{0}\PY{p}{)}
        
        \PY{k}{for} \PY{n}{i} \PY{o+ow}{in} \PY{n+nb}{range}\PY{p}{(}\PY{n}{DIM}\PY{p}{)}\PY{p}{:}
            \PY{n}{A\PYZus{}rhsD}\PY{p}{[}\PY{n}{i}\PY{p}{]} \PY{o}{=} \PY{o}{\PYZhy{}}\PY{n}{AevolParen\PYZus{}dD}\PY{p}{[}\PY{n}{i}\PY{p}{]}
            \PY{k}{for} \PY{n}{j} \PY{o+ow}{in} \PY{n+nb}{range}\PY{p}{(}\PY{n}{DIM}\PY{p}{)}\PY{p}{:}
                \PY{k}{for} \PY{n}{k} \PY{o+ow}{in} \PY{n+nb}{range}\PY{p}{(}\PY{n}{DIM}\PY{p}{)}\PY{p}{:}
                    \PY{n}{A\PYZus{}rhsD}\PY{p}{[}\PY{n}{i}\PY{p}{]} \PY{o}{+}\PY{o}{=} \PY{n}{LeviCivitaDDD}\PY{p}{[}\PY{n}{i}\PY{p}{]}\PY{p}{[}\PY{n}{j}\PY{p}{]}\PY{p}{[}\PY{n}{k}\PY{p}{]}\PY{o}{*}\PY{p}{(}\PY{n}{uU}\PY{p}{[}\PY{n}{j}\PY{p}{]}\PY{o}{/}\PY{n}{u0}\PY{p}{)}\PY{o}{*}\PY{n}{BU}\PY{p}{[}\PY{n}{k}\PY{p}{]}
                    
        \PY{n}{psi6Phi\PYZus{}rhs} \PY{o}{=} \PY{o}{\PYZhy{}}\PY{n}{xi}\PY{o}{*}\PY{n}{alpha}\PY{o}{*}\PY{n}{psi6Phi}
        \PY{k}{for} \PY{n}{j} \PY{o+ow}{in} \PY{n+nb}{range}\PY{p}{(}\PY{n}{DIM}\PY{p}{)}\PY{p}{:}
            \PY{n}{psi6Phi\PYZus{}rhs} \PY{o}{+}\PY{o}{=} \PY{o}{\PYZhy{}}\PY{n}{PevolParenU\PYZus{}dD}\PY{p}{[}\PY{n}{j}\PY{p}{]}\PY{p}{[}\PY{n}{j}\PY{p}{]}
\end{Verbatim}

    \section{\texorpdfstring{Step 9: Build the expression for
\(\tilde{S}_i\)}{Step 9: Build the expression for \textbackslash{}tilde\{S\}\_i}}\label{step-9-build-the-expression-for-tildes_i}

\[\label{step9}\]

{[}Back to Section \ref{top}{]}

We will now find the densitized Poynting flux given by equation 21 in
\href{https://arxiv.org/pdf/1704.00599.pdf}{the original paper},
\[\boxed{\tilde{S}_i = \gamma_{ij} \frac{(v^j+\beta^j)\sqrt{\gamma}B^2}{4 \pi \alpha}.}\]
This is needed to set initial data for \(\tilde{S}_i\) after \(B^i\) is
set from the initial \(A_i\). Note, however, that this expression uses
the drift velocity \(v^i = \alpha v^i_{(n)} - \beta^i\); substituting
this into the definition of \(\tilde{S}_i\) yields an expression in
terms of the Valencia velocity:
\[\tilde{S}_i = \gamma_{ij} \frac{v^i_{(n)} \sqrt{\gamma}B^2}{4 \pi}.\]

    \begin{Verbatim}[commandchars=\\\{\}]
{\color{incolor}In [{\color{incolor} }]:} \PY{c+c1}{\PYZsh{} Step 9: Build the expression for \PYZbs{}tilde\PYZob{}S\PYZcb{}\PYZus{}i}
        \PY{n}{StildeD} \PY{o}{=} \PY{n}{ixp}\PY{o}{.}\PY{n}{zerorank1}\PY{p}{(}\PY{p}{)}
        \PY{n}{BU} \PY{o}{=} \PY{n}{ixp}\PY{o}{.}\PY{n}{declarerank1}\PY{p}{(}\PY{l+s+s2}{\PYZdq{}}\PY{l+s+s2}{BU}\PY{l+s+s2}{\PYZdq{}}\PY{p}{)} \PY{c+c1}{\PYZsh{} Reset the values in BU so that the C code accesses the gridfunctions directly}
        \PY{n}{B2} \PY{o}{=} \PY{n}{sp}\PY{o}{.}\PY{n}{sympify}\PY{p}{(}\PY{l+m+mi}{0}\PY{p}{)}
        \PY{k}{for} \PY{n}{i} \PY{o+ow}{in} \PY{n+nb}{range}\PY{p}{(}\PY{n}{DIM}\PY{p}{)}\PY{p}{:}
            \PY{k}{for} \PY{n}{j} \PY{o+ow}{in} \PY{n+nb}{range}\PY{p}{(}\PY{n}{DIM}\PY{p}{)}\PY{p}{:}
                \PY{n}{B2} \PY{o}{+}\PY{o}{=} \PY{n}{gammaDD}\PY{p}{[}\PY{n}{i}\PY{p}{]}\PY{p}{[}\PY{n}{j}\PY{p}{]} \PY{o}{*} \PY{n}{BU}\PY{p}{[}\PY{n}{i}\PY{p}{]} \PY{o}{*} \PY{n}{BU}\PY{p}{[}\PY{n}{j}\PY{p}{]}
        \PY{k}{for} \PY{n}{i} \PY{o+ow}{in} \PY{n+nb}{range}\PY{p}{(}\PY{n}{DIM}\PY{p}{)}\PY{p}{:}
            \PY{n}{StildeD}\PY{p}{[}\PY{n}{i}\PY{p}{]} \PY{o}{=} \PY{l+m+mi}{0}
            \PY{k}{for} \PY{n}{j} \PY{o+ow}{in} \PY{n+nb}{range}\PY{p}{(}\PY{n}{DIM}\PY{p}{)}\PY{p}{:}
                \PY{n}{StildeD}\PY{p}{[}\PY{n}{i}\PY{p}{]} \PY{o}{+}\PY{o}{=} \PY{n}{gammaDD}\PY{p}{[}\PY{n}{i}\PY{p}{]}\PY{p}{[}\PY{n}{j}\PY{p}{]} \PY{o}{*} \PY{p}{(}\PY{n}{ValenciavU}\PY{p}{[}\PY{n}{j}\PY{p}{]}\PY{p}{)}\PY{o}{*}\PY{n}{sp}\PY{o}{.}\PY{n}{sqrt}\PY{p}{(}\PY{n}{gammadet}\PY{p}{)}\PY{o}{*}\PY{n}{B2}\PY{o}{/}\PY{l+m+mi}{4}\PY{o}{/}\PY{n}{M\PYZus{}PI}
\end{Verbatim}

    \section{Step 10: NRPy+ Module Code
Validation}\label{step-10-nrpy-module-code-validation}

\[\label{step10}\]

{[}Back to Section \ref{top}{]}

Here, as a code validation check, we verify agreement in the SymPy
expressions for the \(\texttt{GiRaFFE}\) evolution equations and
auxiliary quantities we intend to use between 1. this tutorial and 2.
the NRPy+
\href{../edit/GiRaFFE_HO/GiRaFFE_Higher_Order.py}{GiRaFFE\_Higher\_Order.py}
module.

    \begin{Verbatim}[commandchars=\\\{\}]
{\color{incolor}In [{\color{incolor} }]:} \PY{c+c1}{\PYZsh{} Reset the list of gridfunctions, as registering a gridfunction}
        \PY{c+c1}{\PYZsh{}   twice will spawn an error.}
        \PY{n}{gri}\PY{o}{.}\PY{n}{glb\PYZus{}gridfcs\PYZus{}list} \PY{o}{=} \PY{p}{[}\PY{p}{]}
        
        \PY{k}{print}\PY{p}{(}\PY{l+s+s2}{\PYZdq{}}\PY{l+s+s2}{vvv Ignore the minor warning below. vvv}\PY{l+s+s2}{\PYZdq{}}\PY{p}{)}
        
        \PY{c+c1}{\PYZsh{} import BSSN.BSSN\PYZus{}RHSs as bssnrhs}
        \PY{c+c1}{\PYZsh{} bssnrhs.BSSN\PYZus{}RHSs()}
        
        \PY{k+kn}{import} \PY{n+nn}{GiRaFFE\PYZus{}HO.GiRaFFE\PYZus{}Higher\PYZus{}Order} \PY{k+kn}{as} \PY{n+nn}{gho}
        \PY{n}{gho}\PY{o}{.}\PY{n}{GiRaFFE\PYZus{}Higher\PYZus{}Order}\PY{p}{(}\PY{p}{)}
        
        \PY{k}{print}\PY{p}{(}\PY{l+s+s2}{\PYZdq{}}\PY{l+s+s2}{\PYZca{}\PYZca{}\PYZca{} Ignore the minor warning above. \PYZca{}\PYZca{}\PYZca{}}\PY{l+s+se}{\PYZbs{}n}\PY{l+s+s2}{\PYZdq{}}\PY{p}{)}
        
        \PY{k}{print}\PY{p}{(}\PY{l+s+s2}{\PYZdq{}}\PY{l+s+s2}{Consistency check between GiRaFFE\PYZus{}Higher\PYZus{}Order tutorial and NRPy+ module: ALL SHOULD BE ZERO.}\PY{l+s+s2}{\PYZdq{}}\PY{p}{)}
        
        \PY{k}{print}\PY{p}{(}\PY{l+s+s2}{\PYZdq{}}\PY{l+s+s2}{alphau0 \PYZhy{} gho.alphau0 = }\PY{l+s+s2}{\PYZdq{}} \PY{o}{+} \PY{n+nb}{str}\PY{p}{(}\PY{n}{alphau0} \PY{o}{\PYZhy{}} \PY{n}{gho}\PY{o}{.}\PY{n}{alphau0}\PY{p}{)}\PY{p}{)}
        \PY{k}{print}\PY{p}{(}\PY{l+s+s2}{\PYZdq{}}\PY{l+s+s2}{alpsqrtgam \PYZhy{} gho.alpsqrtgam = }\PY{l+s+s2}{\PYZdq{}} \PY{o}{+} \PY{n+nb}{str}\PY{p}{(}\PY{n}{alpsqrtgam} \PY{o}{\PYZhy{}} \PY{n}{gho}\PY{o}{.}\PY{n}{alpsqrtgam}\PY{p}{)}\PY{p}{)}
        \PY{k}{print}\PY{p}{(}\PY{l+s+s2}{\PYZdq{}}\PY{l+s+s2}{AevolParen \PYZhy{} gho.AevolParen = }\PY{l+s+s2}{\PYZdq{}} \PY{o}{+} \PY{n+nb}{str}\PY{p}{(}\PY{n}{AevolParen} \PY{o}{\PYZhy{}} \PY{n}{gho}\PY{o}{.}\PY{n}{AevolParen}\PY{p}{)}\PY{p}{)}
        \PY{k}{print}\PY{p}{(}\PY{l+s+s2}{\PYZdq{}}\PY{l+s+s2}{psi6Phi\PYZus{}rhs \PYZhy{} gho.psi6Phi\PYZus{}rhs = }\PY{l+s+s2}{\PYZdq{}} \PY{o}{+} \PY{n+nb}{str}\PY{p}{(}\PY{n}{psi6Phi\PYZus{}rhs} \PY{o}{\PYZhy{}} \PY{n}{gho}\PY{o}{.}\PY{n}{psi6Phi\PYZus{}rhs}\PY{p}{)}\PY{p}{)}
        \PY{k}{for} \PY{n}{i} \PY{o+ow}{in} \PY{n+nb}{range}\PY{p}{(}\PY{n}{DIM}\PY{p}{)}\PY{p}{:}
        
            \PY{k}{print}\PY{p}{(}\PY{l+s+s2}{\PYZdq{}}\PY{l+s+s2}{uD[}\PY{l+s+s2}{\PYZdq{}}\PY{o}{+}\PY{n+nb}{str}\PY{p}{(}\PY{n}{i}\PY{p}{)}\PY{o}{+}\PY{l+s+s2}{\PYZdq{}}\PY{l+s+s2}{] \PYZhy{} gho.uD[}\PY{l+s+s2}{\PYZdq{}}\PY{o}{+}\PY{n+nb}{str}\PY{p}{(}\PY{n}{i}\PY{p}{)}\PY{o}{+}\PY{l+s+s2}{\PYZdq{}}\PY{l+s+s2}{] = }\PY{l+s+s2}{\PYZdq{}} \PY{o}{+} \PY{n+nb}{str}\PY{p}{(}\PY{n}{uD}\PY{p}{[}\PY{n}{i}\PY{p}{]} \PY{o}{\PYZhy{}} \PY{n}{gho}\PY{o}{.}\PY{n}{uD}\PY{p}{[}\PY{n}{i}\PY{p}{]}\PY{p}{)}\PY{p}{)}
            \PY{k}{print}\PY{p}{(}\PY{l+s+s2}{\PYZdq{}}\PY{l+s+s2}{uU[}\PY{l+s+s2}{\PYZdq{}}\PY{o}{+}\PY{n+nb}{str}\PY{p}{(}\PY{n}{i}\PY{p}{)}\PY{o}{+}\PY{l+s+s2}{\PYZdq{}}\PY{l+s+s2}{] \PYZhy{} gho.uU[}\PY{l+s+s2}{\PYZdq{}}\PY{o}{+}\PY{n+nb}{str}\PY{p}{(}\PY{n}{i}\PY{p}{)}\PY{o}{+}\PY{l+s+s2}{\PYZdq{}}\PY{l+s+s2}{] = }\PY{l+s+s2}{\PYZdq{}} \PY{o}{+} \PY{n+nb}{str}\PY{p}{(}\PY{n}{uU}\PY{p}{[}\PY{n}{i}\PY{p}{]} \PY{o}{\PYZhy{}} \PY{n}{gho}\PY{o}{.}\PY{n}{uU}\PY{p}{[}\PY{n}{i}\PY{p}{]}\PY{p}{)}\PY{p}{)}
            \PY{k}{print}\PY{p}{(}\PY{l+s+s2}{\PYZdq{}}\PY{l+s+s2}{PevolParenU[}\PY{l+s+s2}{\PYZdq{}}\PY{o}{+}\PY{n+nb}{str}\PY{p}{(}\PY{n}{i}\PY{p}{)}\PY{o}{+}\PY{l+s+s2}{\PYZdq{}}\PY{l+s+s2}{] \PYZhy{} gho.PevolParenU[}\PY{l+s+s2}{\PYZdq{}}\PY{o}{+}\PY{n+nb}{str}\PY{p}{(}\PY{n}{i}\PY{p}{)}\PY{o}{+}\PY{l+s+s2}{\PYZdq{}}\PY{l+s+s2}{] = }\PY{l+s+s2}{\PYZdq{}} \PY{o}{+} \PY{n+nb}{str}\PY{p}{(}\PY{n}{PevolParenU}\PY{p}{[}\PY{n}{i}\PY{p}{]} \PY{o}{\PYZhy{}} \PY{n}{gho}\PY{o}{.}\PY{n}{PevolParenU}\PY{p}{[}\PY{n}{i}\PY{p}{]}\PY{p}{)}\PY{p}{)}
            \PY{k}{print}\PY{p}{(}\PY{l+s+s2}{\PYZdq{}}\PY{l+s+s2}{Stilde\PYZus{}rhsD[}\PY{l+s+s2}{\PYZdq{}}\PY{o}{+}\PY{n+nb}{str}\PY{p}{(}\PY{n}{i}\PY{p}{)}\PY{o}{+}\PY{l+s+s2}{\PYZdq{}}\PY{l+s+s2}{] \PYZhy{} gho.Stilde\PYZus{}rhsD[}\PY{l+s+s2}{\PYZdq{}}\PY{o}{+}\PY{n+nb}{str}\PY{p}{(}\PY{n}{i}\PY{p}{)}\PY{o}{+}\PY{l+s+s2}{\PYZdq{}}\PY{l+s+s2}{] = }\PY{l+s+s2}{\PYZdq{}} \PY{o}{+} \PY{n+nb}{str}\PY{p}{(}\PY{n}{Stilde\PYZus{}rhsD}\PY{p}{[}\PY{n}{i}\PY{p}{]} \PY{o}{\PYZhy{}} \PY{n}{gho}\PY{o}{.}\PY{n}{Stilde\PYZus{}rhsD}\PY{p}{[}\PY{n}{i}\PY{p}{]}\PY{p}{)}\PY{p}{)}
            \PY{k}{print}\PY{p}{(}\PY{l+s+s2}{\PYZdq{}}\PY{l+s+s2}{A\PYZus{}rhsD[}\PY{l+s+s2}{\PYZdq{}}\PY{o}{+}\PY{n+nb}{str}\PY{p}{(}\PY{n}{i}\PY{p}{)}\PY{o}{+}\PY{l+s+s2}{\PYZdq{}}\PY{l+s+s2}{] \PYZhy{} gho.A\PYZus{}rhsD[}\PY{l+s+s2}{\PYZdq{}}\PY{o}{+}\PY{n+nb}{str}\PY{p}{(}\PY{n}{i}\PY{p}{)}\PY{o}{+}\PY{l+s+s2}{\PYZdq{}}\PY{l+s+s2}{] = }\PY{l+s+s2}{\PYZdq{}} \PY{o}{+} \PY{n+nb}{str}\PY{p}{(}\PY{n}{A\PYZus{}rhsD}\PY{p}{[}\PY{n}{i}\PY{p}{]} \PY{o}{\PYZhy{}} \PY{n}{gho}\PY{o}{.}\PY{n}{A\PYZus{}rhsD}\PY{p}{[}\PY{n}{i}\PY{p}{]}\PY{p}{)}\PY{p}{)}
            \PY{k}{print}\PY{p}{(}\PY{l+s+s2}{\PYZdq{}}\PY{l+s+s2}{StildeD[}\PY{l+s+s2}{\PYZdq{}}\PY{o}{+}\PY{n+nb}{str}\PY{p}{(}\PY{n}{i}\PY{p}{)}\PY{o}{+}\PY{l+s+s2}{\PYZdq{}}\PY{l+s+s2}{] \PYZhy{} gho.StildeD[}\PY{l+s+s2}{\PYZdq{}}\PY{o}{+}\PY{n+nb}{str}\PY{p}{(}\PY{n}{i}\PY{p}{)}\PY{o}{+}\PY{l+s+s2}{\PYZdq{}}\PY{l+s+s2}{] = }\PY{l+s+s2}{\PYZdq{}} \PY{o}{+} \PY{n+nb}{str}\PY{p}{(}\PY{n}{StildeD}\PY{p}{[}\PY{n}{i}\PY{p}{]} \PY{o}{\PYZhy{}} \PY{n}{gho}\PY{o}{.}\PY{n}{StildeD}\PY{p}{[}\PY{n}{i}\PY{p}{]}\PY{p}{)}\PY{p}{)}
\end{Verbatim}

    \subsubsection{\texorpdfstring{Output to \(\LaTeX\), then
PDF}{Output to \textbackslash{}LaTeX, then PDF}}\label{output-to-latex-then-pdf}

    \begin{Verbatim}[commandchars=\\\{\}]
{\color{incolor}In [{\color{incolor} }]:} \PY{o}{!}jupyter nbconvert \PYZhy{}\PYZhy{}to latex \PYZhy{}\PYZhy{}template latex\PYZus{}nrpy\PYZus{}style.tplx Tutorial\PYZhy{}GiRaFFE\PYZus{}Higher\PYZus{}Order.ipynb
        \PY{o}{!}pdflatex \PYZhy{}interaction\PY{o}{=}batchmode Tutorial\PYZhy{}GiRaFFE\PYZus{}Higher\PYZus{}Order.tex
        \PY{o}{!}pdflatex \PYZhy{}interaction\PY{o}{=}batchmode Tutorial\PYZhy{}GiRaFFE\PYZus{}Higher\PYZus{}Order.tex
        \PY{o}{!}pdflatex \PYZhy{}interaction\PY{o}{=}batchmode Tutorial\PYZhy{}GiRaFFE\PYZus{}Higher\PYZus{}Order.tex
        \PY{c+c1}{\PYZsh{}!rm \PYZhy{}f Tut*.out Tut*.aux Tut*.log}
\end{Verbatim}

    \begin{Verbatim}[commandchars=\\\{\}]
{\color{incolor}In [{\color{incolor} }]:} 
\end{Verbatim}


    % Add a bibliography block to the postdoc
    
    
% \bibliographystyle{unsrt}
% \bibliography{ipython}
% 
    
    \end{document}
